\begin{definition}
	Sei $H$ ein Hilbertraum sowie \(H_1, H_2\) Unterhilbertr\as ume. Dann bez. \(U\in \B(H)\) \textit{partielle Isometrie} von $H_1$ nach $H_2$, wenn 
	\[U^*U = \text{Pr}_{H_1} \;\text{ und }\; UU^* = \text{Pr}_{H_2}\;.\]
	Das hei\ss{}t konkret:
	\begin{align*}&\forall x\in H_1: \norm{Ux} = \norm{x},\;\;\forall x \in H_2: \norm{U^*x} = \norm{x},\\
		&\forall x \in H_1^{\perp}: Ux = 0,\;\; \forall x \in H_2^\perp: U^*x = 0\;.\end{align*}
\end{definition}
\begin{theorem}
	Sei $H$ ein Hilbertraum, \(A\in \B(H)\). Dann existiert eine partielle Isometrie von \(\ker(A)^\perp = \ker(\ab A)\) nach \(\text{im}(A)\), sodass \(A = U\cdot \ab{A}\).
\end{theorem}
\unsure{Wieso gilt \(\ker(\ab{A} = \ker (A)^\perp\)?}
\begin{rem}
	Dies ist analog zur Polarzerlegung bei komplexen Zahlen, jedoch gilt i. A. 
	\[\C\ni z = \ab z \exp(i\varphi(z))\implies \ab{\overline z} = \ab z \;\text{ vs }\; T \in \B(H) \implies \ab{T^*} = \sqrt{TT^*} \neq \sqrt{T^*T} = \ab{T}\;.\]
\end{rem}
\begin{proof}
	Es gilt f\us r alle \(x\in H\)
	\[\norm{Ax}^2 = \ip{Ax, Ax} = \ip{x, A^*Ax} = \ip{x, \ab{A}^2 x} = \ip{\ab{A}x, \ab{A}x} = \norm{\ab{A}x}^2\;.\]
	Damit ist \(K:=\ker(A)^\perp\) ein Hilbertraum mit Norm 
	\[\norm x_A := \norm {\ab{A}x}^2\]
	und \(\ab{A}: K\to H\) ist eine Isometrie.\ldots
\end{proof}