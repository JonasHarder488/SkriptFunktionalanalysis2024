\begin{definition}
	Sei $H$ ein Hilbertraum sowie \(H_1, H_2\) Unterhilbertr\as ume. Dann bez. \(U\in \B(H)\) \textit{partielle Isometrie} von $H_1$ nach $H_2$, wenn 
	\[U^*U = \text{Pr}_{H_1} \;\text{ und }\; UU^* = \text{Pr}_{H_2}\;.\]
	Das hei\ss{}t konkret:
	\begin{align*}&\forall x\in H_1: \norm{Ux} = \norm{x},\;\;\forall x \in H_2: \norm{U^*x} = \norm{x},\\
		&\forall x \in H_1^{\perp}: Ux = 0,\;\; \forall x \in H_2^\perp: U^*x = 0\;.\end{align*}
\end{definition}
\begin{theorem}
	Sei $H$ ein Hilbertraum, \(A\in \B(H)\). Dann existiert eine partielle Isometrie \(U\in \B(H)\) von \(\ker(A)^\perp \) nach \(\text{im}(A)\), sodass \(A = U \ab{A}\).
	\label{part_isom}
\end{theorem}
\unsure{In der Vorlesung war \(\ker(\ab{A}) = \ker (A)^\perp\) gegeben, ist dies richtig? (Werner S. 297: \(\ker(\ab{A}) = (\text{im}(\ab{A})^\perp)\))}
\begin{rem}
	Dies ist analog zur Polarzerlegung bei komplexen Zahlen, jedoch gilt i. A. 
	\[\C\ni z = \ab z \exp(i\varphi(z))\implies \ab{\overline z} = \ab z \;\text{ vs }\; T \in \B(H) \implies \ab{T^*} = \sqrt{TT^*} \neq \sqrt{T^*T} = \ab{T}\;.\]
\end{rem}
\begin{proof}[Beweis]
	Es gilt f\us r alle \(x\in H\)
	\[\norm{Ax}^2 = \ip{Ax, Ax} = \ip{x, A^*Ax} = \ip{x, \ab{A}^2 x} = \ip{\ab{A}x, \ab{A}x} = \norm{\ab{A}x}^2\;.\]
	Damit ist \(K:=\ker(A)^\perp\) ein Hilbertraum mit Norm 
	\[\norm x_A := \norm {\ab{A}x}^2\]
	und \(\ab{A}: K\to H\) ist eine Isometrie. Wir definieren
	\[U: \text{im}(\ab A) = \ker(\ab A)^\perp \to \text{im} (A),\;\; U(\ab A x ) := Ax\;.\]
	Damit gilt
	\[\ab A x = \ab A y \implies (x-y) \in \ker \ab A \implies x-y \in \ker A \implies Ax = Ay\;.\]
	Dabei haben wir ausgenutzt, dass wegen $\ab A$ selbstadjungiert \(\ker(\ab A) = \ker(A)\) \us ber
	\[\norm{\ab A x}^2 = \ip{x, \ab A^2 x} = \ip{x, A^*Ax}= \norm{Ax}^2\;. \]
	Setze weiterhin f\us r \(y\in \ker(\ab A)\)  \(U(y):=0\) und damit ist $U$ wohldefiniert und eine partielle Isometrie.
\end{proof}

\begin{theorem}
	Sei $T$ Spurklasse-Operator auf einem Hilbertraum $H$. Dan ist auch $T^*$ ein Spurklasse-Operator. Au\ss{}erdem gilt \(\tr(T^*) = \overline{\tr(T)}\).
\end{theorem}
\begin{proof}[Beweis]
	Nach Satz \ref{part_isom} existiert eine partielle Isometrie \(U\in\B(H): \ker(T)^\perp = \text{im}(\ab T) \to \text{im}(T)\) mit
	\[T = U\ab{T} \implies T^* = \ab{T}U^*\;.\]
	Somit gilt 
	\[TT^* = U\ab{T}^2 U^* = U\underbrace{\ab{T}\overbrace{U^* U}^{=\text{Pr}_{\text{im}(\ab{T})}} \ab{T}}_{=\ab{T}^2} U^* \implies \ab{T^*} = U\ab{T} U^*\;.\]
	Dabei erhalten wir die letzte Folgerung \us ber: 
	\[\ab{T}^2 = \ab T U^* U \ab{T} \implies \ab{T^*}^2 = U\ab{T}\ab{T}U^* = (U\ab{T}U^*)^2\;.\]
	Wir w\as hlen nun eine ONB \(\seq en\), sodass \(\forall n \in\N:\) \(U^* e_n = f_n\) oder \(U^* e_n = 0\). Somit gilt
	\[\ip{f_n, f_m} = \ip{U^* e_n, U^* e_m} = \ip{e_n, e_m} \;.\]
	Dies macht also auch \(\seq f n\) zu einem ONS, sei \(((f_n)_{n\in\N\cup S})\) ONB. Dabei haben wir ausgenutzt, dass mit \(U^*:  \text{im}(T) \to \ker(T)^\perp\) nach Definition \(UU^* = \text{Pr}_{\text{im}(T)}\) gilt.\\
	\unsure{M\us sste f\us r die Definition von $U^*$ nicht \(H = \text{im}(T) \oplus \text{im}(T)^\perp\) gelten? Dies gilt aber nur f\us r \(\text{im}(T)\) abgeschlossen, was i. A. nicht gilt.}
	Damit ergibt sich nun:
	\[\sum_{n\in\N}\ip{e_n, \ab{T^*} e_n} = \sum_{n\in\N}\ip{e_n, U\ab T U^* e_n} = \sum_{n\in\N}\ip{U^* e_n, \ab T U^* e_n} \leq \sum_{n\in\N\cup S} \ip{f_n, \ab T f_n} < \infty\;.\]
	Somit ist \(T^*\) in der Tat ein Spurklasse-Operator. Weiterhin gilt
	\[\tr(T^*) = \sum_{n\in\N} \ip{e_n, T^* e_n} = \sum_{n\in\N}\ip{T e_n, e_n} = \sum_{n\in\N}\ip{Te_n, e_n} = \sum_{n\in\N} \overline{\ip{e_n, T e_n}} = \overline{\sum_{n\in\N}\ip{e_n, Te_n}} = \overline{\tr(T)}\;.\]
\end{proof}

\begin{rem}
	Sei \(\varrho\in \B(H)_{\geq 0}\) mit \(\tr(\varrho)<\infty\). Dann gilt f\us r alle \(T \in \B(H)\), dass \(T\varrho\) ein Spurklasse-Operator ist, d. h. die Abbildung \(T\mapsto \tr(T\varrho)\) ist ein lineares Funktional.
\end{rem}

\begin{theorem}
	Sei $H$ ein Hilbertraum, \(\varrho \in \B(H)_{\geq 0}\) mit \(\tr(\varrho) < \infty\). Dann gilt
	\[\forall T\in \B(H): \ab{\tr(T\varrho)} \leq \tr(\varrho) \norm{T}\;.\]
	Sei weiterin \(\seq Tn \subseteq \B(H)^\N\) mit \(0\leq T_n\leq T_{n+1} \leq T\in \B(H)\) gegeben, dann gilt
	\[\tr\left(\left(\sup_{n\in\N}T_n\right)\varrho\right) = \sup_{n\in\N} \tr(T_n\varrho)\;.\] 
\end{theorem}

\begin{proof}[Beweis]
	Nach Satz \ref{spurklasse_geq0_Folge} existiert \(\seq \lambda n \subseteq \R^\N\) mit 
	\[\varrho = \sum_{n\in\N} \lambda_n | e_n \rangle \langle e_n |, \;\; \forall n\in\N: \lambda_n \geq 0, \;\;\sum_{n\in\N} \lambda_n = \tr(\varrho) < \infty\;.\]
	Damit gilt
	\[T\varrho = \sum_{n\in\N} \lambda_n T | e_n \rangle \langle e_n | = \sum_{n\in\N} \lambda_n | Te_n \rangle \langle e_n | \;.\]
	Somit erhalten wir die erste Behauptung
	\[\tr(T\varrho) = \sum_{n\in\N} \langle e_n | T\varrho | e_n \rangle = \sum_{n\in\N} \lambda_n \ip{e_n, Te_n} \implies \ab{\tr(T\varrho)} \leq \sum_{n\in\N} \lambda_n \ab{\ip{e_n, Te_n}} \leq \sum_{n\in\N} \lambda_n\norm T\;.\]
	Sei nun \(\seq Tm\) wie oben angegeben, \(T_m \overset{m\to\infty}{\longrightarrow} \sup_{m\in\N}T_m\). Mit Satz \textit{[...]} folgt:
	\unsure{Label fehlt (Satz 8.20 in Vorlesung)}
	\[\ip{e_n, T_m e_n} \overset{m\to\infty}{\longrightarrow} \ip{e_n, \sup_{m\in\N} T_m e_n}\;.\]
	Anwendung des Satzes von Beppo-Levi liefert die zweite Behauptung:
	\[\tr(T_m\varrho)\overset{m\to\infty}{\longrightarrow}  \sum_{n\in\N} \lambda_n\ip{e_n, \sup_{m\in\N} T_m e_n}\;.\]
\end{proof}
\begin{rem}
	Es besteht eine Analogie zwischen dem zweiten Teil des Satzes sowie der $\sigma$-Additivit\as t nichtkommutativer Ma\ss{}e. 
\end{rem}

\section{Der Spektralsatz und der messbare Funktionalkalk\us l}
\begin{rem}
	Ziel des messbaren Kalk\us ls ist es eine funktionalanalytische Beziehung zwischen stetigen Funktionen und der Ma\ss{}- und Integrationstheorie herzustellen.
\end{rem}

\begin{theorem}[Riesz-Markov-Kakutani]
	Sei $K$ ein kompakter metrischer Raum, $\Sigma$ die Borelmengen auf $K$. Dann gibt es eine Bijektion zwischen $C^0(K)$ und 
	\[M(K) := \set{\mu:\Sigma \to \C\;\vert\; \mu \text{ (endliches) signiertes Ma\ss{}}}\]
	mittels \(\mu(f):= \int f d\mu\). Weiterhin ist $\mu$ endliches Ma\ss{} \(\iff \forall f \geq 0 : \mu(f) \geq 0\).
\end{theorem}

\begin{theorem}
	Sei $K$ ein kompakter metrischer Raum, $\Sigma$ die Borelmengen auf $K$ sowie \(L^\infty(K):= \set{f: K\to\C\;\vert\; f \text{ messbar, beschr\as nkt}}\). Sei weiterhin \(V\subseteq L^\infty(K)\) mit den Eigenschaften \label{lemma_mess_kalk}
	\begin{enumerate}
		\item \(C^0(K)\subseteq V\)
		\item F\us r alle \(\seq fn \subseteq V^\N\) mit \(\forall t \in K: \li fn(t) := f(t)\) existiert sowie  \(\sup_{n\in\N}\norms{f_n} < \infty\) gilt \(f\in V\). \label{lemma_mess_kalk_2}
	\end{enumerate}
	Dann gilt \(V = L^\infty(K)\).
\end{theorem}

\begin{proof}[Beweis]
	Wir definieren f\us r \(f\in L^\infty(K)\)
	\[V_f := \set{g\in L^\infty(K)\;\vert\; f+g \in V}.\]
	Sei \(f\in C^0(K)\), dann \(C^0(K)\subseteq V_f\). Sei weiterhin eine Folge \(\seq gn \subseteq V_f^\N\) gegeben, mit \(\sup_{n\in\N} \norms{g_n} < \infty\) und einem existierenden punktweisen Grenzwert \(\li gn = g\). Damit folgt sofort \(g\in V_f\). Somit \(V_f = V\), also 
	\[f \in C^0(K), g\in V \implies f_0 + g \in V\;.\]
	Sei nun \(f\in V\), dann gilt ebenfalls \(C^0(K) \subseteq V_f\) und wir folgern analog \(V_f = V\), also 
	\[f\in V, g\in V \implies f + g \in V\;.\] 
	Somit ist $V$ also abgeschlossen bzgl. der Addition. Genauso l\as sst sich \[\alpha \in \C, g\in V \implies \alpha g \in V\] zeigen, was $V$ zu einem \(\C\)-Vektorraum macht.
	Sei nun \(G\subseteq K\) offen, 
	\[1_G(x) := \lime{n} \overbrace{\min(1,n\cdot \text{dist}(x, G^C))}^{\in C^0(K)} \implies 1_G \in V\;.\]
	Man kann leicht zeigen:
	\[\seq An,\;\; A_n \in \Sigma \text{ paarweise disjunkt} \implies 1_{\bigcup_{n\in\N} A_n} = \sum_{n\in\N} 1_{A_n}\]
	Weiterhin gilt
	\[1_A\in V \implies 1_{A^C} = \overbrace{1}^{\in C^0(K)}-1_A\]
	Aus letzteren beiden Relationen folgt: \(\set{A\in \Sigma \;\vert\; 1_A\in V} = \Sigma\). Mit Eigenschaft \ref{lemma_mess_kalk_2} folgt dann die Behauptung.
\end{proof}

\begin{rem}[Erweiterung Funktionalkalk\us l]
	Sei \(T\in \B(H)\) selbstadjungiert, \(f\in C^0(\sigma(T))\) sowie \(x,y \in H\) mit $H$ Hilbertraum. Definiere:
	\[l_{x,y}(f):= \ip{f(T)x,y}\]
	Damit ist \(l_{x,y}:C^0(\sigma(T)) \to \C\) linear und stetig mit
	\[\ab{l_{x,y}(f)} \leq \norm{f(T)} \norm x\norm y = \norms f \norm x \norm y\;.\]
	Dann existiert genau ein Ma\ss{} $\mu_{x,y}$ auf \(\sigma(T)\) mit 
	\[\forall f \in C(\sigma(T)): \ip{f(T)x,y} = \int_{\sigma(T)} f(z) \;\mu_{x,y}(dz)\;\;(*)\;.\]
	Weiterhin gilt \(\norm{\mu_{x,y}}_{TV} \leq \norm x\norm y\) wobei $\normn{\cdot}{TV}$ die \textit{Totalvariationsnorm}:
	\[\normn{\mu}{TV} = \sup\set{\ab{\int fd\mu}\;\Big\vert\; \forall x \in K: \ab{f(x)} \leq 1, \;f \in L^\infty(K)[\text{bzw.} f\in C^0(K)]}\;.\]
	Die rechte Seite von $(*)$ ist auch f\us r $f\in L^\infty(\sigma(T))$ sinnvoll, wodurch wir auch f\us r solche $f$ ein $f(T)$ definieren wollen (messbarer Kalk\us l).
	\label{rem_mess_kalk}
\end{rem}

\begin{theorem}[Messbarer Funktionalkalk\us l]
	Sei \(T\in \B(H)\) normal. Dann gibt es genau eine Abbildung \(\varphi:L^\infty(\sigma(T)) \to \B(H)\) mit \label{messb_kalk}
	\begin{enumerate}
		\item \(\varphi(\text{id}) = T\), \(\varphi(1) = \I\), wobei \(\forall x\in \sigma(T): \; \text{id}(x) = x, \;1(x) = 1\)
		\item \(\forall f\in L^\infty(\sigma(T)): \varphi(\overline f) = \varphi(f)^*\) \label{eigenschaft_mess_kalk_2}
		\item \(\forall f, g \in L^\infty(\sigma(T))\;\forall \lambda \in \C: \varphi(\lambda f + g) = \lambda \varphi(f) + \varphi(g)\)\; und \; \(\varphi(fg) = \varphi(f) \varphi(g)\)\label{eigenschaft_mess_kalk_3}
		\item \(\forall f\in L^\infty(\sigma(T)): \norm{\varphi(f)} = \norms f\) \label{eigenschaft_mess_kalk_4}
		\item Falls \(\forall n\in\N: f_n \in L^\infty(\sigma(T)),\;\exists c>0\;\forall n\in\N: \norms{f_n} \leq c\) und \(\forall t \in \sigma(T): \lime{n}{f_n(t)} = f(t)\), so gilt \(\forall x \in H: \lime{n}{\varphi(f_n)x} = \varphi(f)x\). \label{eigenschaft_mess_kalk_5}
	\end{enumerate}
\end{theorem}

\begin{proof}[Beweis]
	\textbf{Schritt 1} (Eindeutigkeit) Sei \(\tilde\varphi\) eine weitere solche Abbildung. Wir nutzen Satz \ref{lemma_mess_kalk} mit
	\[V = \set{f\in L^\infty(\sigma(T))\;\vert\; \varphi(f) = \tilde\varphi(f)}\;.\]
	Nach Voraussetung wissen wir, dass \(\text{id} \in V,\; 1 \in V\). Aufgrund der Eindeutigkeit des stetigen Funktionalkalk\us ls gilt \(C^0(\sigma(T)) \subseteq V\). Wegen Eigenschaft \ref{eigenschaft_mess_kalk_5} ist Satz \ref{lemma_mess_kalk} anwendbar und wir erhalten \(V = L^\infty(\sigma(T)) \iff \varphi = \tilde \varphi\).\\
	\textbf{Schritt 2} (Existenz) Analog zu Bem. \ref{rem_mess_kalk} definieren wir f\us r \(f\in L^\infty(\sigma(T))\) die stetige Sesquilinearform
	\[u: H\times H \to \C,\;\; u(x,y) = \int_{\sigma(T)} f(z) \mu_{x,y}(dz)\;.\]
	Damit ist $u$ insbesondere beschr\as nkt, da
	\[\ab{u(x,y)} \leq \norms f \norm{\mu_{x,y}} \leq \norms f \norm x\norm y\;.\]
	Nach Satz \ref{sesq_darstellung} existiert somit in der Tat \(\varphi(f) \in \B(H)\) mit
	\[u(x,y) = \ip{x,\varphi(f)y}\;.\]
	\textbf{Schritt 3} (Eigenschaften) \textit{(Zu \ref{eigenschaft_mess_kalk_2}:)} Seien \(x,y\in H\), dann gilt
	\begin{align*}
		\ip{x,\varphi(\overline f)y} &= \int_{\sigma(T)} \overline{f}(z) \mu_{x,y}(dz)  = \overline{\int_{\sigma(T)} f(z) \overline{\mu_{x,y}}(dz)} = \overline{\int_{\sigma(T)} f(z) \mu_{y,x}(dz)} \\&= \overline{\ip{y, \varphi(f)x}} = \ip{\varphi(f)x,y} \implies \varphi(\overline f) =\varphi(f)^*
	\end{align*}
	\textit{(Zu \ref{eigenschaft_mess_kalk_3}:)} Beweis analog\\
	\textit{(Zu \ref{eigenschaft_mess_kalk_5}:)} Seien \(f_n, f \in L^\infty(\sigma(T))\) wie gefordert. Damit wissen wir nach dem Satz von Lebesgue \us ber die majorisierte Konvergenz:
	\[\forall x, y \in H:\; \lime{n}{\int_{\sigma(T)} f_n(z) \mu_{x,y}(dz)} = \int_{\sigma(T)} f(z) \mu_{x,y}(dz)\;\text{ und }\;\lime{n}{\ip{x,\varphi(f_n)y}} = \ip{x, \varphi(f)y}\;.\]
	Sei nun \(x\in H\) beliebig, so gilt
	\begin{align*}
		\norm{\varphi(f_n)x}^2 &= \ip{\varphi(f_n)x, \varphi(f_n)x} = \ip{x, \varphi(f_n)^*\varphi(f_n)x} = \ip{x, \varphi(\overline{f_n})\varphi(f_n)x} \\
		&= \ip{x,\varphi\left(\ab{f_n}^2\right)x} \overset{n\to\infty}{\longrightarrow} \ip{x, \varphi(\ab{f}^2)x} = \norm{\varphi(f)x}^2\;.
	\end{align*}
	Deshalb gilt
	\begin{align*}
		&\norm{\varphi(f_n)x -\varphi(f)x}^2 = \ip{\varphi(f)x, \varphi(f)x} - 2\Re(\ip{\varphi(f_n)x, \varphi(f)x}) + \ip{\varphi(f_n)x, \varphi(f_n)x}\\
		& \overset{n\to\infty}{\longrightarrow} \ip{\varphi(f)x, \varphi(f)x} -2 \ip{\varphi(f)x, \varphi(f)x} + \ip{\varphi(f)x, \varphi(f)x} = 0\;.
	\end{align*}
	\happybegin
	Dies war gerade zu zeigen.\hfill\happyend
\end{proof}

\begin{definition}
	Sei \(T\in \B(H)\) normal. Dann ist das zugeh\os rige \textit{Spektralma\ss{}} \(E^T: \Sigma(\C) \to \B(H)\) definiert als
	\[E^T(A) = 1_{A\cap \sigma(T)}(T) = \varphi(1_A)\]
	Dabei bez. \(f(T) = \varphi(f)\) f\us r alle \( f \in L^\infty(\sigma(T))\).
\end{definition}

\begin{theorem}[Eigenschaften Spektralma\ss{}]
	Es gilt f\us r \(T\in \B(H)\) normal:
	\end{theorem}
	\begin{enumerate}
		\item \(\forall A \in \Sigma(\C):\) \(E^T(A)\) ist eine orthogonale Projektion. \label{eigenschaften_spektralmass_1}
		\item \(\forall A, B \in \Sigma(\C):\) \(E^T(A\cap B) = E^T(A)E^T(B) = E^T(B)E^T(A)\)\label{eigenschaften_spektralmass_2}
		\item F\us r alle \(\seq An \subseteq \Sigma(\C)^\N\), wobei die $A_n$ paarweise disjunkt, \(\forall x \in H\) gilt \label{eigenschaften_spektralmass_3}
		\[\sum_{n\in\N}E^T(A_n)x = E^T\left(\bigcup_{n\in\N} A_n\right)x\;.\]
		\item \(\sigma(T)\) ist die kleinste kompakte Menge $K$ mit \(E^T(K)= \I\).\label{eigenschaften_spektralmass_4}
	\end{enumerate}
	\begin{proof}[Beweis]
		\textbf{(Zu \ref{eigenschaften_spektralmass_1})} Eine orthogonale Projektion $P\in\B(H)$ erf\us llt die Eigenschaften $P= P^2 = P^*$. Es gilt 
		\[\forall A \in \Sigma(\C): 1_A^2 = 1_a \implies \varphi(1_A) = \varphi(1_A^2) = \varphi(1_A) \varphi(1_A)  \iff E^T(A) = (E^T(A))^2\;.\]
		Analog l\as sst sich folgern
		\[\overline{1_A} = 1_A \implies E^T(A)^* = E^T(A)\;.\]
		\textbf{(Zu \ref{eigenschaften_spektralmass_2})} Die Behauptung folgt aus 
		\[\forall A, B \in \Sigma(\C): 1_A1_B = 1_{A\cap B}\;.\]
		\textbf{(Zu \ref{eigenschaften_spektralmass_3})} Dies folgt direkt aus Satz \ref{messb_kalk}, (\ref{eigenschaft_mess_kalk_5}).\\
		\textbf{(Zu \ref{eigenschaften_spektralmass_4})} Nach \ref{messb_kalk} gilt \(E^T(\sigma(T)) = \I\). Sei nun 
		\[K \subseteq \sigma(T) \;\text{ mit } \; E^T(K) = \I \implies E^T(K^C) = 0\;.\]
		Somit gilt 
		\[\forall f \in L^\infty(\sigma(T)) \text{ mit } f\vert_K = 0:\; \varphi(f) = 0 \implies f\vert_{\sigma(T)} = 0 \implies f=0 \implies K = \sigma(T)\;.\]
	\end{proof}
	
	\begin{definition}
		Ein \textit{projektionswertiges Ma\ss{}} (auf $\C$, $H$ Hilbertraum) ist eine Abbildung \(E:\Sigma(\C)\to \B(H)\) mit:
		\begin{enumerate}
			\item \(\forall A \in \Sigma(\C):\) \(E(A) = E(A)^2 = E(A)^*\)
			\item \(\forall A, B \in \Sigma(\C):\) \(E(A)E(B) = E(A\cap B)\)
			\item \(E(\emptyset) = 0\), \(E(\C) = \I\)
			\item \(\forall \seq An \subseteq \Sigma(\C)^\N\), $A_n$ paarweise disjunkt, \(\forall x\in H:\)
			\[\sum_{n\in\N}E(A_n)x = E\left(\bigcup_{n\in\N}A_n\right)x\;\text{ mit }\; E(K) = \I \text{ f\us r kompaktes }K\subseteq \C\;.\]
		\end{enumerate}
	\end{definition}
	
	\begin{theorem}[Spektralsatz f\us r normale, beschr\as nkte Operatoren]
		Sei $E$ ein projektionswertiges Ma\ss{}. Dann existiert genau ein normales \(\B(H)\ni T =: \int_{\sigma(T)} z E(dz)\) mit
		\[\forall f \in L^\infty(\C)\;\forall x, y\in H:\; \ip{x,\varphi(f)y} = \int_{\sigma(T)} f(z)\ip{x,E(dz)y}\;.\]
	\end{theorem}
	\unsure{Wieso geht hier \(f\in L^\infty(\C)\) und nicht \(f\in L^\infty(\sigma(T))\)?}
		\begin{proof}[Beweis]
			Sei $K\subseteq \C$ kompakt. Setze
			\unsure{Ich vermute \(K \subseteq \C\) kompakt, in der Vorlesung wurde dies jedoch nicht angegeben}
			\[u(x,y) = \int_{\C} z \;1_K(z) \ip{x, E(dz)y} \implies \ab{(u(x,y))} \leq \max\set{z\;\vert\;z\in K}\norm x\norm y\]
			Nach \ref{sesq_darstellung} existiert \(T\in \B(H)\) mit \(u(x,y) = \ip{x,Ty}\). Somit gilt 
			\[\forall f \in L^\infty(K):\; \ip{x, f(T)y} = \int_K f(z) \ip{x, E(dz) y}\;.\]
			Dabei ist die Abbildung 
			\[\Sigma(\C) \supseteq A \mapsto \ip{x, E(A)y}\]
			ein Ma\ss{}, dasselbe Ma\ss{}, welches in Bem. \ref{rem_mess_kalk} mit \(\mu_{x,y}\) bezeichnet wurde.
		\end{proof}
		\begin{rem}[Integration bzgl. eines projektionswertigen Ma\ss{}es]
			Zur Integration einer beschr\as nkten, messbaren Funktion $f$ gehen wir in Schritten analog zur skalaren Integration vor.\\
			\textbf{1. Schritt: Indikatorfunktionen} (\(f = 1_A,\; A\in\Sigma(\R)\)). Dann setze
			\[\int f(z) E(dz) = E(A)\;.\]
			\textbf{2. Schritt: Treppenfunktionen} (\(f = \sum_{i=1}^n \alpha_i 1_{A_i}\)) Dann setze
			\[\int f(z) E(dz) = \sum_{i=1}^n \alpha_i E_{A_i}\;.\]
			\textbf{3. Schritt:} Sei $f$ beschr\as nkt und messbar, dann existiert eine Folge \(\seq fn\) von Treppenfunktionen, die gleichm\as \ss{}ig gegen $f$ konvergiert. Es l\as sst sich zeigen, dass gerechtfertigt ist
			\[\int f dE = \lime{n}{\int f_n dE}\]
			zu setzen.
		\end{rem}
		
		\begin{theorem}
			Sei \(T\in \B(H)\)normal, sowie \(f: \sigma(T)\to \C\), \(g:\C \to\C\) beschr\as nkt und messbar. Dann gilt \label{komposition_mess_kalk}
			\[(f\circ g)(T) = f(g(T))\;.\]
		\end{theorem}
		\begin{proof}[Beweis]
			Wir sehen, dass \(E^{g(T)}(A) = E^T(g^{-1}(A))\) gilt (intuitiv wegen \(1_A \circ g = 1_{g^{-1}(A)}\)). Somit:
			\[f(g(T)) = \int f(z) E^{g(T)}(dz) = \int f(z)E^T(g^{-1}(dz)) = \int f(g(z)) E^T(dz) = (f\circ g)(T)\;.\]
		\end{proof}
		\begin{theorem}
			Sei $H$ Hilbertraum, \(T\in \B(H)\) normal mit Spektralma\ss{} \(E^T\). Dann gilt 
			\begin{enumerate}
				\item \(\lambda \in \varrho(T) \iff \exists \varepsilon > 0:\; E^T(K_\varepsilon(\lambda)) = 0\) \label{spektralmass_spektrum_1}
				\item \(\lambda \in \C\) ist Eigenwert von $T$ \(\iff E^T(\set{\lambda}) \neq 0\). In diesem Fall ist \(E^T(\set\lambda)\) die Projektion auf \(\ker(T-\lambda\I)\). \label{spektralmass_spektrum_2}
				\item Isolierte Punkte \(\lambda \in \sigma(T)\) (d. h. \(\exists \varepsilon > 0:\; K_\varepsilon(\lambda)\cap \sigma(T) = \set{\lambda}\)) sind Eigenwerte von $T$. \label{spektralmass_spektrum_3}
			\end{enumerate}
		\end{theorem}
		\begin{proof}[Beweis]
			\textbf{(Zu \ref{spektralmass_spektrum_1}.)} Nach Konstruktion gilt offensichtlich \(E^T(\varrho(T)) = 0\). Zeige nun die R\us ckrichtung. Sei also eine solches \(\varepsilon\) gegeben. Setze
			\[f(z) = \frac{1}{\lambda - z} \cdot1_{K_\varepsilon(\lambda)^C}(z)\]
			Dann ist $f$ messbar und auf \(\sigma(T)\) beschr\as nkt. Dasselbe gilt f\us r \(g(z) = \lambda - z\). Somit (mit Satz \ref{komposition_mess_kalk}) 
			\[f(T)(\lambda -T) = f(T)g(T) = (fg)(T) = 1_{K_\varepsilon(\lambda)^C}(T) = E^T(K_\varepsilon(\lambda)^C) = \I\;.\]
			Dabei haben wir \(\I = E^T(\C) = E^T(K_\varepsilon(\lambda)^C) + E^T(K_\varepsilon(\lambda))\) ausgenutzt. Analog zeigt man \((\lambda-T)f(T) = \I\) und somit gilt in der Tat \(\lambda \in \varrho(T)\).\\
			\textbf{(Zu \ref{spektralmass_spektrum_2}.)} Sei \(\lambda \in \C\) Eigenwert zu \(x\in H\), d. h. \(Tx = \lambda x\). Aus \ref{stetiger_kalkuel} ist bekannt, dass f\us r $f$ stetig \(f(T)x = f(\lambda)x\) gilt. Mit Satz \ref{lemma_mess_kalk} folgt, dass diese Gleichung auch f\us r beschr\as nkte, messbare $f$ gilt. Somit
			\[E^T(\set\lambda)x = 1_{\set\lambda}(T)x = 1_{\set\lambda}(\lambda)x = x\;.\]
			Sei nun andersherum \(E^T(\set\lambda)\neq 0\). Sei nun \(x\in \text{im}(E^T(\set\lambda))\), d. h. \(x = E^T(\set\lambda) x\). Somit
			\[\ip{y, (\lambda -T)x} = \ip{y, (\lambda-T)E^T(\set\lambda)x} = \int(\lambda-z)1_{\set\lambda}(z)\ip{y, E^T(dz)x} = 0\;.\]
			Damit in der Tat \(x\in \ker(\lambda - T)\).\\
			\textbf{(Zu \ref{spektralmass_spektrum_3}.)} W\as hle \(\varepsilon > 0\) mit \(K_\varepsilon(\lambda)\cap\sigma(T) = \set\lambda\). Nach dem \ref{spektralmass_spektrum_1}. Punkt des Satzes gilt \[K_\varepsilon(\lambda)\setminus\set\lambda\subseteq \varrho(T)\implies E^T(K_\varepsilon(\lambda)\setminus\set\lambda) = 0\]
			W\as hre nun \(E^T(\set\lambda) = 0\), dann auch \(E^T(K_\varepsilon(\lambda)) = 0\) und deshalb \(\lambda\in \varrho(T)\). Somit \(E^T(\set\lambda) \neq 0\) und folglich Eigenwert. 
		\end{proof}
	
