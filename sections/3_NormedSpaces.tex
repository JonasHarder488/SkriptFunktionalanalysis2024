\chapter{Normierte R\as ume}
\section{Definitionen}
\begin{rem}
	Wir betrachten hier Vektorr\as ume \us ber den K\os rpern \(\K = \R\) oder \(\K = \C\).
\end{rem}
\begin{definition}
	Eine \textit{Norm} \us ber einem $\K$-Vektorraum $V$ ist eine Abbildung \(\norm \cdot: V \to \R_{\geq 0}\) mit:
	\begin{enumerate}[noitemsep]
		\item \(\forall x \in V, \lambda \in \K: \norm{\lambda x} = \ab \lambda \norm x\) \label{skalar_norm}
		\item \(\forall x, y \in V: \norm{x+y} \leq \norm x + \norm y\)
		\item \(\norm x  = 0 \iff x = 0\) \label{definit_norm}
	\end{enumerate} 
	Dann hei\s t \(V, \norm \cdot\) normierter Raum.
\end{definition}
\begin{rem}
	Falls \ref{definit_norm}. nicht gilt, bezeichnen wir die Abbildung als \textit{Halbnorm}.
\end{rem}

\begin{theorem}
	Ein normierter Raum \((V, \norm\cdot)\) ist ein metrischer Raum mit der Metrik $d$, definiert durch
	\[\forall x, y \in V: d(x,y) = \norm{x-y}\;.\]
\end{theorem}
\begin{ex}
	In diesem Fall gilt f\us r die Operationen \(+: V\times V \to V\) und \(\cdot: \K\times V \to V\) (mit \(\lambda \in \K, \; x, y, x, y' \in V\)):
	\begin{align*}
		& d(x' + y', x + y) = \norm{x' + y' - (x+y)} \leq \norm{x' + y'} + \norm{x+y} \\
		&d(\lambda x, \lambda x') = \norm{\lambda(x-x')} = \ab{\lambda} \norm{x-x'} = \ab{\lambda}d(x,x')
	\end{align*}
\end{ex}

\begin{definition}
	Ein \textit{Banachraum} ist ein vollst\as ndiger normierter Raum.
\end{definition}

\section{Vervollst\as ndigung}

\begin{theorem}
	Sei \((V,\norm\cdot)\) ein normierter Raum, dann existiert eine Vervollst\as ndigung \((\hat V, \hnorm\cdot)\), d. h. $V$ kann in einen Banachraum eingebettet werden.
\end{theorem}
\begin{proof} 
	Definiere Analog zu Satz \ref{vervollst_mR}:
	\begin{align*}
		&\hat V_0 := \{\seq x n \vert \forall n \in \N: x_n \in V, \; \seq x n \text{ Cauchyfolge}\}.\\
	&\text{\As quivalenzrelation $\sim$ auf $\hat V_0$: } 
	\seq xn \sim \seq yn :\iff \lim_{n\to\infty} \norm{x_n, y_n} = 0\\
	&\text{Menge aller \As quivalenzklassen: }
	\hat V = \faktor{\hat V_0}{\sim} = \{\equ{\seq xn} \vert \seq xn \in \hat X_0\} 
	\end{align*}
	Dabei ist \(\hat V\) ein Vektorraum mit \(\forall \lambda \in \K, \seq xn, \seq yn\in \hat V_0:\)
	\[\lambda \equ{\seq xn} := \equ{\seq{\lambda x}{n}}\text{ und } \equ{\seq xn} + \equ{\seq yn} = \equ{\se{x_n + y_n}{n}}\;.\]
	Als Norm auf \(\hat V\) definieren wir 
	\[\hnorm{\equ{\seq xn}} = \lime n \norm{x_n}\;.\]
	Zeige zun\as chst die Wohldefiniertheit. Der obige Grenzwert existiert, da f\us r \(\seq xn \in \hat V_0\) die Folge \(\se{\norm{x_n}}{n}\) eine Cauchyfolge ist, mit
	\[\lim_{n,m \to\infty} \ab{\norm{x_n} - \norm x_m} \leq \lim_{n,m\to\infty}\norm{x_n - x_m} = 0 \;.\]
	Betrachte nun \(\seq xn, \seq yn \in \hat V_0, \; \seq xn \sim \seq yn\), dann
	\[\lime n\ab{\norm{x_n} - \norm{y_n}} \leq \lime n \norm{x_n -y_n}  = 0 \]
	und somit ist \(\hnorm \cdot\) unabh\as ngig vom Repr\as sentanten. F\us r die Normeigenschaften zeige hier nur die Dreiecksungleichung und Definitheit (\ref{skalar_norm}. Eigenschaft trivial):
	\begin{multline*}\hnorm{\equ{\seq xn} + \equ{\seq yn}} = \lime n \norm{x_n + y_n} \leq \lime n \norm{x_n} + \norm{y_n} \\= \lime n \norm{x_n} + \lime n \norm{y_n} = \hnorm{\equ{\seq xn}} + \hnorm{ \equ{\seq yn}}\end{multline*}
	Weiterhin:
		\[\hnorm{\equ{\seq xn}} = 0 \iff \lime n \norm{x_n} = 0 \iff \seq xn \sim \se{0}{n} \iff \equ{\seq xn} = \equ{\se{0}{n}} = 0_{\hat V}\]
\end{proof}

\section{$L^p$-R\as ume}

\begin{definition}
	Sei \((\Omega, \mc F, \mu)\) ein Ma\s raum, \(p\in [1,\infty)\), dann definieren wir
	\[\p^p(\Omega, \mc F, \mu) := \set{f:\Omega \to \mathbb{K} \vert f\in \mc M(\Omega, \mc F), \; \int_\Omega \ab{f}^p < \infty}\;.\]
	Wir schreiben kurz auch \(\p^p(\mu)\).
\end{definition}
\begin{rem}
	\(\p^p(\Omega, \mc F, \mu)\) definiert einen $\K$-Vektorraum.
\end{rem}

\begin{theorem}
	\(\normn\cdot p: \p^p(\mu) \to \K, \; \normn fp := \left(\int_\Omega \ab{f}^p d \mu\right)^{\sfrac{1}{p}}\) ist eine Halbnorm. \label{p_halbnorm}
\end{theorem}
\begin{proof}
	Siehe nach Satz \ref{Hoelder}.
\end{proof}
\begin{rem}
	\(\normn\cdot p\) ist keine Norm.
\end{rem}
\begin{ex}
	Betrachte \(\Omega = \R,\; f = 1_C \neq 0 ,\; \mu = l\), wobei $C$ die Cantormenge und $l$ das Lebesgue-Ma\s{} sind. Dann gilt:
	\[\normn fp = \left(\int_\Omega (1_C)^p dl\right)^\frac 1p = l(C)^{\frac 1p} = 0\]
	
\end{ex}
\begin{lemma}[Youngsche Ungleichung]
	Seien \(u,v \in \R_{\geq 0}, \; p, q \in \R_{>1}, \; \frac{1}{p} + \frac{1}{q} = 1\). Dann gilt
	\[u\cdot v \leq \frac{u^p}{p} + \frac{v^q}{q}\;.\]\label{Young}
\end{lemma}

\begin{theorem}[H\os lder Ungleichung]
	Sei \((\Omega, \mc F, \mu)\) ein Ma\s raum, \(p, q \in (1,\infty)\), \(\frac{1}{p} + \frac{1}{q} = 1\) und \(f\in \p^p(\mu), \;g\in \p^q(\mu)\). Dann gilt \(fg \in \p^1(\mu)\) und \(\normn{fg}{1} \leq \normn fp \normn gq\).
	\label{Hoelder}
\end{theorem}
\begin{proof}
	Nehmen o B. d. A. an \(f,g \geq 0, \; \normn fp = \normn gq = 1\). Mit Lemma \ref{Young} gilt
	\begin{multline*}\normn{fg}{1} = \int_\Omega f(x) g(x) \mu(dx) \leq \int_\Omega \frac{f(x)^p}{p} + \frac{g(x)^q}{q}\mu(dx) = \frac{1}{p} \int_\Omega f(x)^p \mu(dx) + \frac{1}{q} \int_\Omega g(x)^q \mu(dx) \\= \frac{1}{p} + \frac{1}{q} = 1 = \normn fp \cdot \normn gq
	\end{multline*}
\end{proof}

\begin{proof}[\textbf{Zu Satz \ref{p_halbnorm}}]
	\ref{skalar_norm}. Eigenschaft ist trivial.
	Zeige nun noch die Dreiecksungleichung. Seien daf\us r \(f,g \in \p^p(\mu)\). Dann gilt auch die Absch\as tzung:
	\[\ab{f+g}^p = \ab{f+g}^{p-1} \ab{f+g} \leq \ab{f+g}^{p-1} \ab{f} + \ab{f+g}^{p-1} \ab{g}  \;.\] 
	Sei \(q \in \R\) so gew\as hlt, dass \(\frac{1}{p} + \frac{1}{q} = 1 \iff p+q = pq\). Dann gilt \(\ab{f+g}^{p-1} \in \p^q(\mu)\), da 
	\[\left(\ab{f+g}^{p-1}\right)^q = \ab{f+g}^{pq - q} = \ab{f+g}^{p}\]
	Dies ist in der Tat integrierbar, da \(\ab{f+g}^p \leq \ab{f}^p + \ab{g}^p\) und nach Voraussetzung \(f,g \in \p^p(\mu)\). Somit erhalten wir 
		\[\normn{f+g}{p}^p = \int_\Omega \ab{f+g}^p  d\mu\leq  \int_\Omega  \ab{f+g}^{p-1} \ab{f} + \ab{f+g}^{p-1} \ab{g} d \mu = \normn{(f+g)^{p-1} f}{1} + \normn{(f+g)^{p-1} g}{1} \;.\]
		Anwenden der H\os lder Ungleichung ergibt
		\[\leq \normn{f+g}{q}\normn{f}{p} + \normn{f+g}{q}\normn{g}{p} = \left(\int_\Omega\ab{f+g}^{(p-1)q}\right)^\frac{1}{q} (\normn{f}{p}+\normn{g}{p}) =  \left(\int_\Omega\ab{f+g}^{p}\right)^\frac{1}{q} (\normn{f}{p}+\normn{g}{p})\;.\]
		Sei o. B. d. A. \(f+g \neq 0\) fast \us berall (sonst Beh. trivial), dann gilt: 
		\[\normn{f+g}{p}^p \leq \normn{f+g}{p}^{\sfrac{p}{q}}(\normn{f}{p}+\normn{g}{p}) \iff\normn{f+g}{p}^{p (1-\sfrac{1}{q})} \leq\normn{f}{p}+\normn{g}{p} \iff  \normn{f+g}{p} \leq \normn{f}{p}+\normn{g}{p}\]
\end{proof}

\begin{definition}
	Sei \(\Omega, \mc F, \mu\) ein Ma\s raum sowie \(f, g: \Omega \to \K\). Dann definieren wir
	\[f \muq g :\iff \mu(\set{\omega \in \Omega\vert f(\omega) \neq g(\omega)} =: \mu(\set{f\neq g}) = 0\;.\]
	Man sagt auch \(f = g\) $\mu$-fast \us berall.
\end{definition}
\begin{rem}
	Die Relation \(\muq\) ist eine \As quvialenzrelation.
\end{rem}
\begin{theorem}
	Sei \((\Omega, \mc F, \mu)\) ein Ma\s raum sowie \(f\in \p^p(\mu), \; g:\Omega \to \K\). Dann gilt
	\[f \muq g \implies g \in \p^p(\mu) \text{ und } \normn fp = \normn gp\;.\]
	\label{Lp_Norm_wohldef}
\end{theorem}
\begin{proof}
	Gelte o. B. d. A. \(g \muq 0\).Betrachte hier \(g:\Omega \to \K \cup\set{-\infty, \infty}\), sei \(A:=\set{\omega \in \Omega\vert g(\omega) \neq 0}\). Nach Voraussetzung gilt \(\mu(A) = 0\) und somit
	\[\ab{g} \leq \infty \cdot 1_A \implies \int_\Omega \ab{g}^p d\mu \leq \infty \cdot \int_\Omega 1_A d\mu = \infty\cdot 0 = 0\;.\]
	Folglich gilt also in der Tat \(g\in \p^p\). Zeige nun \(\normn fp = \normn gp\). Dabei nutzen wir \(f\muq g \iff g-f \muq 0 \implies \normn{g-f}{p} = 0\). Es folgt
	\[\normn gp = \normn{f+(g-f)}{p} \leq \normn fp + \normn{g-f}{p} = \normn fp\]
	Analog erhalten wir \(\normn fp \leq \normn gp\) und somit in der Tat \(\normn fp = \normn gp\).
\end{proof}

\begin{theorem} Sei \((\Omega, \mc F, \mu)\) ein Ma\s raum, dann wird
	\(\faktor{\p^p(\mu)}{\muq}\) mit \(\forall f \in \p^p(\mu): \;\normn{\equ f}{p} := \normn fp\) ein normierter Raum.
\end{theorem}
\begin{proof}
	Die Wohldefiniertheit der Norm folgt aus Satz \ref{Lp_Norm_wohldef}. Weiterhin l\as ssen sich die Halbnormeigenschaften auf Satz \ref{p_halbnorm} zur\us ckf\us hren. Es g. z. z., dass \(\normn \cdot p\) auf \(\faktor{\p^p(\mu)}{\muq}\) definit ist. Angenommen f\us r \(\equ{f} \in \faktor{\p^p(\mu)}{\muq}\)  gilt \(\normn{\equ{f}}{p} = 0 \iff \int_\Omega \ab{f}^p = 0\). Nehme weiterhin an, dass \(f \overset{\mu}{\neq} 0\iff 0< \mu(\set{\ab{f}>0})\). Somit
	\[0<\mu(\set{\ab{f}>0}) = \mu\left(\bigcup_{n\in\N}\set{\ab{f} > \frac{1}{n}}\right) = \lime n\mu \left(\ab{f}> \frac{1}{n}\right)\;.\]
	Folglich \(\exists c>0: \mu(\set{\ab{f}\geq c}) > 0\). Damit erhalten wir mit \(A := \set{\ab{f}>c}\)
	\[\ab{f}^p \geq c^p \cdot 1_A \implies \int \ab{f}^p \geq c^p \mu(A) > 0\;.\]
	Dies ist ein Widerspruch und folglich gilt \(f\muq 0\).
\end{proof}

\begin{definition}
	Sei (\(\Omega, \mc F, \mu\)) ein Ma\s raum, dann 
	\[L^p(\mu) := L^p(\Omega,\mc F \mu) := \faktor{\p^p(\Omega, \mc F,\mu)}{\muq}\]
	Wir schreiben meist \(L^p(\mu)\) f\us r \(\K = \R\) und \(L^p(\mu, \C)\) f\us r \(\K = \C\).
\end{definition}
 \begin{rem}
 	Statt \(\equ{f} \in L^p\) schreiben wir nur \(f\in L^p\).
 \end{rem}
\begin{definition}
	\(f\in \mc M(\Omega, \mc F, \mu)\) hei\s t \textit{wesentlich beschr\as nkt} (\(:\iff f \in \p^\infty(\mu)\)), falls
	\[\exists c \in \R_{>0} : \mu(\set{\omega \in\Omega\vert \ab{f(\omega)} > c}) = 0\;.\]
	Dann definieren wir
	\[\norms f := \inf\set{c\;\vert\;\mu(\set{\omega \in\Omega\vert \ab{f(\omega)} > c}) = 0}\;.\]
\end{definition}

\begin{theorem}
	\(\norms\cdot\) ist eine Halbnorm und es gilt \(f\muq g \iff \norms{f-g} = 0\).
\end{theorem}
\begin{proof}
	Die \ref{skalar_norm}. Eigenschaft und die zweite Aussage sind trivial, wir zeigen hier wieder nur die Dreiecksungleichung. Seien \(f,g \in \mc M(\Omega, \mc F, \mu)\) wesentlich beschr\as nkt, \(c,d > 0\) und \(A := \set{\omega \in \Omega\;\vert\; \ab{f(\omega)} > c}\), \(B : = \set{\omega\in\Omega\;\vert\;\ab{g(\omega)} > d}\). Somit 
	\[\mu(A) = \mu(B) = 0 \implies \mu(A\cup B) \leq \mu(A) + \mu(B) = 0\;.\]
	Es gilt 
	\[D: = \set{\omega\in\Omega\;\vert\;\ab{f(\omega) + g(\omega)} > c+d} \subseteq A\cup B \]
	 und somit auch \(\mu(D) = 0\).
	Sei nun \(\omega \in (A\cup B)^C\), dann \[\ab{f(\omega)} \leq c \land \ab {g(\omega)} \leq d \implies \ab{f(\omega) + g(\omega)} \leq \ab{f(\omega)} + \ab{g(\omega)} \leq c+d\;.\]
	D. h. \(f+g\) ist wesentlich durch \(c+d\) beschr\as nkt und somit folgt in der Tat
	\[\norms{f+g} \leq c+d \leq \norms{f} + \norms{g}\;.\]
\end{proof}

\begin{theorem}
	Sei \(\Omega, \mc F, \mu\) ein Ma\s raum. Dann ist \(\faktor{\p^\infty(\mu)}{\muq}\) mit \(\forall f \in \p^\infty(\mu): \norms{\equ{f}} = \norms{f}\) ein normierter Raum.
\end{theorem}

\begin{definition}
	Sei \((\Omega,\mc F, \mu)\) ein Ma\s raum, dann 
	\[L^\infty(\mu) := L^\infty(\Omega,\mc F, \mu) := \faktor{\p^\infty(\Omega, \mc F, \mu)}{\muq}\;.\]
	Wir schreiben meist \(L^\infty(\mu)\) f\us r \(\K = \R\) und \(L^\infty(\mu, \C)\) f\us r \(\K = \C\).
\end{definition}

\begin{rem}
	\(L^p(\mu)\) f\us r \(p \in [1,\infty]\) sind keine direkten Funktionenr\as ume.
\end{rem}

\begin{theorem}
	Sei \((\Omega, \mc F, \mu)\) ein Ma\s raum, dann ist \((L^p(\mu), \normn \cdot p)\) ein Banachraum f\us r \(p \in [1,\infty]\).
\end{theorem}
\begin{proof}
	Betrachte hier nur den Fall \(p = 1\). Sei \(\seq fn\) eine Cauchyfolge in \(L^p(\mu)\), d. h. 
	\[\forall \varepsilon> 0 \;\exists N_0 \in \N \;\forall m,n \geq N_0: \normn{f_m - f_n}{p} < \varepsilon\;.\]
	Daher gilt insbesondere
	\[\forall k > 0 \; \exists n_k \;\forall m \geq n_k : \normn{f_m - f_{n_k}}{p} < 2^{-k}\;.\]
	Somit k\os nnen wir solche \(n_k\) mit \(n_{k+1} > n_k\) w\as hlen und erhalten eine Teilfolge \(\se{f_{n_k}}{k}\) mit \( \normn{f_{n_{k+1}} - f_{n_k}}{p} < 2^{-k}\). F\us r \(p = 1 \) mit Beppo-Levi und \(g_k:= \ab{f_{n_{k+1}} - f_{n_k}}\):
	\[\sum_{k\in\N} \normn{f_{n_{k+1}} - f_{n_k}}{1} = \sum_{k\in\N} \int_\Omega g_k d\mu = \int_\Omega \sum_{k\in\N} g_k d\mu < \infty\;.\]
	Somit gilt insbesondere 
	\[\mu\left(\set{\omega \in \Omega\Big\vert \sum_{k\in\N} g_k(\omega) = \infty}\right) = 0\;.\]
	Nehmen hier o. B. d. A. an, dass \(\forall \omega \in \Omega: \sum_{k\in\N} g_k(\omega) <\infty\). Somit bildet \(\forall \omega\in\Omega\) die Folge \(\se{f_{n_k}}{k}\) eine Cauchyfolge und ist somit konvergent, d. h. \(\forall \omega \in \Omega: f(\omega) := \lime k f_{n_k}(\omega)\) existiert. Somit gilt 
	\[\lime k f_{n_0}(\omega) - f_{n_k}(\omega) = f_{n_0}(\omega) - f(\omega)\;\]
	Wir k\os nnen eine integrierbare Majorante f\us r die $f_{n_0}(\omega) - f_{n_k}(\omega)$ finden mit 
	\[\ab{f_{n_0}(\omega) - f_{n_k}(\omega)} \leq \sum_{l=0}^{k-1}\ab{f_{n_l}(\omega) - f_{n_{l+1}}(\omega)} \leq \sum_{k\in\N} g_k(\omega) =: g(\omega) \text{ mit }\int_\Omega g d\mu < \infty\;.\]
	Nach dem Satz von Lebesgue folgt somit, dass $f$ integrierbar (d. h. \(f\in L^1(\mu)\)) und 
	\[\lime k \int_\Omega \ab{ f_{n_0}(\omega) - f_{n_k}(\omega) -( f_{n_0}(\omega) - f(\omega))} = 0 \iff \lime k \normn{f_{n_k} - f}{1} = 0 \;.\]
	Also ist \(L^1(\mu)\) in der Tat vollst\as ndig.
\end{proof}

\section{Beispiele f\us r normierte R\as ume}

\begin{ex}
	Betrachte 
	\[\Omega = \set{1,\ldots, n}, \; n\in\N_{>0} \;\text{ mit }\; \forall A \subseteq \Omega: \mu(A) = \ab{A}\;.\] 
	Dann gilt wegen der Korrespondenz als Vektorraum
	\[L^p(\mu) = \R^n \text{ und } L^p(\mu, \C) = \C^n\]
	wegen der Korrespondenz 
	\[f:\set{1,\ldots,n} \to \K \;\;\longleftrightarrow\;\; \begin{pmatrix}f(1)\\ \ldots \\ f(n)\end{pmatrix}\;.\]
\end{ex}

\begin{ex}[Folgenr\as ume]
	Folgende Vektorr\as ume 
	\begin{align*}
		 c_0 &:= \set{\seq an \;\vert\; \forall n \in\N: a_n \in \K, \; \li an  = 0}\\
		c &:= \set{\seq an \;\vert\; \forall n\in \N: a_n \in \K, \; \li an \text{ existiert}}\\
	   \ell^\infty = \ell^\infty(\N) &:= \set{\seq an \;\vert\; \forall n\in \N: a_n \in \K,\; \exists c>0\;\forall n\in\N: \ab{a_n} \leq c}
	\end{align*}
	sind normierte Vektorr\as ume (insbesondere Banachr\as ume) mit  der Supremumsnorm 
	\[\norms{\seq an} = \sup_{n\in\N}\ab{a_n}\;.\]
	Betrachte weiterhin die Folgenr\as ume \(l^p\) mit \(p \in [1,\infty)\), wobei $\mu$ das Z\as hlma\s{}:
	\[\ell_\K^p = L^p(\N, \mu,\K) = \set{\seq an \;\Big\vert\; \forall n\in \N: a_n \in \K, \sum_{n=1}^\infty \ab{t_n}^p < \infty}\]
	mit der Norm \(\normn \cdot p\) definiert durch
	\[\forall \seq an \in \ell^p: \normn{\seq an}{p} = \left(\sum_{n=1}^\infty \ab{t_n}^p\right)^\frac{1}{p}\;.\]
	Dabei sind \((\ell^p, \normn \cdot p)\) f\us r \(p \in [1,\infty)\) Banachr\as ume. 
\end{ex}

\begin{ex}
	Betrachte
	\[\Omega = \R  \text{ bzw. } \Omega = \R^d\text{ mit Lebesgue-Ma\s{} } l \text{ bzw. } l^d\;.\]
	mit den entsprechenden R\as umen \(L^p(l), \; L^p(l,\C)\) bzw. \(L^p(l^d)\) und den bekannten Normen.
	Betrachte 
	\[\Omega = [0,1] \text{ mit Lebesgue-Ma\s{} } l \]
	dann ergibt sich 
	\[L^p([0,1],l) \text{ mit } \forall f\in L^p([0,1],l): \normn fp = \int_0^1 \ab{f(x)}^p dx\]
	Der normierte Raum \((C^0([0,1]), \norms \cdot)\) $\ldots$ 
	\end{ex}
	\begin{rem}
		Es gibt eine Inklusion \(C^0([0,1])\xhookrightarrow{} L^p([0,1])\), da in jeder \As quivalenzklasse nur eine stetige Funktion ist.
	\end{rem}