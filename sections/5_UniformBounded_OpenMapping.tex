\begin{definition} \label{nat_einb} 
	%%%Vorlesung vom 27.5%%%
	Sei $V$ ein Vektorraum über dem Körper $\K$, $V^{'}$ der Dualraum und $V^{''}$ der Bidualraum. Dann ist die natürliche Einbettung $\iota$ definiert durch: 
	\[
	\iota: V \to V^{''}, \iota(v)(\varphi) = \varphi(v) \text{ für alle } v \in V, \varphi \in V'
	\]
	Außerdem ist $\iota$ per Definition injektiv und linear. Des Weiteren erhält $\iota$ die Norm:
	\[
	\norm{\iota(v)}_{V^{'}} = \norm{v}_V
	\]
\end{definition}

\begin{definition} 
	Ein normierter Raum $X$ heißt reflexiv, wenn die natürliche Einbettung $X \to X^{''}$ bijektiv (bzw. surjektiv) ist.
\end{definition}

\begin{rem} 
	Sei $X'' = (X')'$ ein Banachraum, dann ist $X$ ein Banachraum. Es gibt damit keine nichtvollständigen reflexiven Räume.
\end{rem}

\begin{ex} 
	Sei $1 < p < \infty$ und $\frac{1}{p}+\frac{1}{q} = 1$, dann gilt $L^p(\mu)' = L^q(\mu)$ für $\sigma$-endliche Maße. Wir sehen aber ebenfalls $L^q(\mu)' = L^p(\mu)$, also ist $L^p(\mu)$ tatsächlich reflexiv.
	
	Am einfachsten ist $p = q = 2$. Also ist $L^2(\mu)$ ein Selbstdual.
\end{ex}

\begin{rem} 
	Wir haben also meistens eine von beiden Ketten:
	\begin{align*}
		& X \xrightarrow{\iota} X' \xrightarrow{\iota'} X'' \simeq X \\
		& X \xrightarrow{\iota} X' \simeq X
	\end{align*}
	Tatsächlich kann es sonst nur unendlich lange Ketten geben. Betrachte das folgende Beispiel:
\end{rem}

\begin{ex} 
	Wir untersuchen nun die Dualkette von 
	\[
	c_0 = \left\{ \{a_n\}_{n \in \mathbb{N}} : a_n \in \mathbb{R}, \lim_{n \to \infty} a_n = 0 \right\}
	\]
	Es stellt sich heraus, dass $c_0' = \ell^1$. Außerdem stellt sich heraus, dass $(\ell^1)' = \ell^{\infty}$ und wiederum $(\ell^{\infty})' = \left\{ \mu : 2^{\mathbb{N}} \to \mathbb{R}, \text{ $\mu $ ist endlich additiv} \right\}$. Bei $\mu$ handelt es sich nicht nur um Maße, sondern um Inhalte. Diese Kette zieht sich so fort.
\end{ex}

\section{Das Prinzip der gleichmäßigen Beschränktheit}
\begin{definition}
	Sei $(X, \rho)$ ein Maßraum, dann heißt $S \subset X$ nirgends dicht, wenn die Vervollständigung $\bar{S}$ keine Kugel enthält. Mit 
	\[
	\bar{S} = \left\{ x \in X : \forall n \in \mathbb{N} \exists x_n \in S: x_n \xrightarrow{n \to \infty} x \right\} \text{ und } \bar{S} = S \cup \partial S.
	\] 
	Das Innere ist definiert als 
	\[
	S^{\mathrm{o}} = \left\{ x \in X \mid \exists \varepsilon > 0: K_{\varepsilon}(X) \subseteq S \right\}.
	\] 
	$S$ ist genau dann nirgends dicht, wenn $\bar{S}^{\mathrm{o}} = \emptyset$.
\end{definition}

\begin{definition}
	Sei $Y \subset X$, $\{S_n\}_{n \in \mathbb{N}}$ eine abzählbare Familie nirgends dichter Mengen, dann heißt $Y$ mager genau dann, wenn $Y = \bigcup_{n \in \mathbb{N}} S_n$.
\end{definition}

\begin{ex}
	Sei $X = \mathbb{R}$. 
	\begin{enumerate}
		\item Die Cantormenge $C$ ist nirgends dicht.
		\item $\mathbb{Q}$ ist nirgends dicht.
	\end{enumerate}
\end{ex}

\begin{lemma} \label{lemma_glm_beschr}
	Seien $(V, \norm{\cdot}_V)$ sowie $(W, \norm{\cdot}_W)$ normierte Räume, $T: V \to W$ eine lineare Abbildung. Sei $r, \varepsilon > 0$, $x_0 \in V$ mit $\overline{K_{\varepsilon}(x_0)} \subset T^{-1}(\overline{K_r(0)})$, dann ist $T$ beschränkt und $\norm{T} \leq \frac{2r}{\varepsilon}$.
	
	\begin{proof}[Beweis:] 
		Sei $y \in V$ und $\norm{y} \leq \varepsilon$, dann $x_0, x_0 + y \in T^{-1}(\overline{K_r(0)})$, also $\norm{Tx_0} \leq r$ und $\norm{T(x_0 + y)} \leq r$. Mit der Dreiecksungleichung folgt: 
		\[
		\norm{Ty} = \norm{T(x_0 + y) - T(x_0)} \leq 2r.
		\] 
		Sei nun $x \in V$, $\norm{x}_V \leq 1$, dann sehen wir 
		\[
		\norm{Tx} = \frac{1}{\varepsilon} \norm{T(\varepsilon x)} \leq 2 \frac{r}{\varepsilon}
		\]
	\end{proof}
\end{lemma}

\begin{rem} Die Kategorien sind wie folgt definiert: 
	\begin{enumerate} 
		\item Eine Menge $M$ heißt von \textit{Kategorie 1}, wenn es eine Folge $\{S_n\}_{n \in \mathbb{N}}$ nirgends dichter Mengen gibt mit $M = \bigcup_{n \in \mathbb{N}} S_n$.
		\item Eine Menge $M$ heißt von \textit{Kategorie 2}, wenn $M$ nicht von Kategorie 1 ist.
	\end{enumerate}
\end{rem}

\begin{theorem}[Cantor'scher Schnitt] \label{Cantor} 
	Sei $(X, d)$ ein vollständiger metrischer Raum, $\{K_n\}_{n \in \mathbb{N}}$ eine absteigende Folge nichtleerer abgeschlossener Mengen in $X$, mit $K_{n+1} \subset K_n$ für alle $n \in \mathbb{N}$. Sei außerdem $\lim_{n \to \infty} \text{diam}(K_n) = 0$, dann gibt es $x \in X$ mit $\bigcap_{n = 1}^{\infty} K_n = \{x\}$.
\end{theorem}

\begin{theorem} \label{baire_kat}
	Sei $(X, \rho)$ ein vollständiger metrischer Raum und $\{S_n\}_{n \in \mathbb{N}}$ eine Folge nirgends dichter Mengen, dann gibt es $x \in X$ mit $x \notin \bigcup_{n \in \mathbb{N}} S_n$.
	
	\begin{proof}[Beweis:] 
		Da $S_n$ nirgends dicht ist, ist $S_n^{\mathrm{o}} = \emptyset$. Nach Definition gibt es zu jedem $S_n$ und jeder offenen Menge $U \subseteq X$ eine Menge $B \subseteq U$ mit $\bar{B} \cap S_n = \emptyset$. Sei $B_1$ nun eine beliebige nicht-leere offene Teilmenge von $X$. Da $S_1$ nirgends dicht ist, gibt es $B_2 \subset B_1$ mit $\bar{B}_2 \cap S_1 = \emptyset$. So definiere man sich eine Folge absteigend nicht-leerer offener Mengen $\{B_n\}_{n \in \mathbb{N}}$ wobei $B_{n+1} \subseteq B_n$ und $\bar{B}_{n+1} \cap S_n = \emptyset$. Da $X$ vollständig ist, garantiert der Satz von Cantor, dass der Durchschnitt der abgeschlossenen Mengen $\bar{B}_n$ nicht leer ist, also $\bigcap_{n = 1}^{\infty} \bar{B}_n \not= \emptyset$. 
	\end{proof}
\end{theorem}

\begin{theorem}[Gleichmäßige Beschränktheit] \label{glm_beschr} \\
	Seien $(V, \norm{\cdot}_V)$ und $(W, \norm{\cdot}_W)$ normierte Räume und $\mathcal{F}$ eine Familie von Operatoren mit $\mathcal{F} \subseteq \B(V, W)$ mit 
	\[
	\forall x \in V: \sup{\left\{\norm{Tx}_W: T \in \mathcal{F}\right\}} < \infty,
	\] 
	dann gilt $\sup{\left\{\norm{T}_{V, W}: T \in \mathcal{F}\right\}} < \infty.$
	
	\begin{proof}[Beweis:] 
		Wir zerlegen $V$ zunächst in abzählbare Mengen mit 
		\[
		S_n := \left\{ x \in V: \forall T \in \mathcal{F}, \norm{Tx} \leq n \right\} 
		\]
		und $V = \bigcup_{n \in \mathbb{N}} S_n$. Aufgrund von \ref{baire_kat} hat $\bar{S}_n$ ein nichtleeres Inneres für ein beliebiges $n \in \mathbb{N}$. $S_n$ ist abgeschlossen: Sei $x_m \in S_n$ mit $m = 1, 2, \ldots$ und $x_m \to x$. Dann gilt für alle $T \in \mathcal{F}$, dass $\norm{Tx_m} \leq n$, mit Stetigkeit aber $\norm{Tx_m} \to \norm{Tx}$. Also $\norm{Tx} \leq n$ für alle $T \in \mathcal{F}$.
		
		Sei nun $n \in \mathbb{N}$ fixiert, dann hat $S_n$ einen inneren Punkt $x_0$. Sei $\varepsilon > 0$ so gewählt, dass 
		\[
		\overline{K_{\varepsilon}(x_0)} \subseteq S_n \subseteq \left\{ x \in V: \norm{Tx} \leq n, T \in \mathcal{F} \right\},
		\]
		dann folgt mit \ref{lemma_glm_beschr} $\norm{T} \leq \frac{2n}{\varepsilon}$.
	\end{proof}
\end{theorem}

\begin{theorem}[Banach-Steinhaus] \\
	Sei $\{T_n\}$ eine Folge beschränkter linearer Operatoren mit $T_n \in \B(X, Y)$ für alle $n \in \mathbb{N}$. Außerdem existiere für alle $x \in X$ auch $\lim_{n \to \infty} T_n x$. Definiere $T: X \to Y$ durch $Tx = \lim_{n \to \infty} T_n x$. Dann ist $T \in \B(X, Y).$
	
	\begin{proof}[Beweis:] 
		$T$ ist offensichtlich linear. Sei $x \in X$, dann ist $\sup_{n}{\norm{T_n x}} < \infty$, da $\{T_n x\}$ nach Voraussetzung konvergiert und konvergente Folgen in metrischen Räumen beschränkt sind. \ref{glm_beschr} liefert uns nun, dass $\sup{\norm{T_n}} = M < \infty$. Damit folgt über die CBS-Ungleichung für alle $x \in X, n \in \mathbb{N}$, dass $\norm{T_n x} \leq M \norm{x}$.
	\end{proof}
\end{theorem}


\section{Das Prinzip der gleichm\as \s igen Beschr\as nktheit}
%%%%%Vorlesung 30.05.2024%%%%%
\begin{theorem}
	Sei \((V,\normn\cdot V)\) ein normierter Raum und \(A\subseteq V\) eine Teilmenge mit
	\[\forall \varphi \in V': \sup\set{\ab{\varphi(x)} \;\big\vert\; x\in A}<\infty\;.\]
	Dann ist A beschr\as nkt.
\end{theorem}
\begin{proof}
	Betrachte die kanonische Einbettung \(\Phi: V \to V'',\; \Phi(x)(\varphi) = \varphi(x)\). Diese ist isometrisch, wie in Satz \ref{Doppeldual_1} gezeigt. Sei nun 
	\[\mc F = \Phi(A) = \set{\Phi(x)\;\vert\;x \in A} \subseteq V'' = \B(V',\K)\;.\]
	Damit gilt nun  f\us r alle \(\varphi\in V'\)
	\begin{align*}
		& \sup \set{\norm{Y(\varphi)} \;\big\vert\; Y\in \Phi(A)} = \sup \set{\ab{Y(\varphi)} \;\big\vert\; Y\in \Phi(A)} \\
		 =& \sup \set{\ab{\varphi(\Phi^{-1}(Y))} \;\big\vert\; Y\in \Phi(A)} =  \sup \set{\ab{\varphi(x)} \;\big\vert\; x\in A} < \infty
	\end{align*}
	Dabei haben wir in der zweiten Zeile \(\Phi^{-1}(\Phi(A)) = A\) (da \(\Phi\) injektiv) ausgenutzt. Nach Satz \ref{B(V,W)_Banach} ist \(V'\) ein Banachraum, das hei\s t wir k\os nnen den Satz zur gleichm\as \s igen Beschr\as nktheit anwenden, mit \(\mc F \) wie oben definiert. Es gilt also
	\begin{multline*} 
		\sup \set{\norm{Y(\varphi)} \;\big\vert\; Y\in \Phi(A)} <\infty\implies  \\
		\infty > \sup \set{\norm{Y}_{V''} \;\big\vert\; Y\in \Phi(A)} =  \sup \set{\norm{\Phi(x)}_{V''} \;\big\vert\; x\in A} =  \sup \set{\norm{x}_V \;\big\vert\; x\in A}\;.
	\end{multline*}
	Somit ist \(A\) in der Tat beschr\as nkt.
\end{proof}
\begin{rem}
	Seien \(V, A\) wie im Satz. Wir bezeichnen die A als schwach beschr\as nkt, falls
	\[\forall \varphi \in V': \sup\set{\ab{\varphi(x)} \;\big\vert\; x\in A}<\infty\;.\]
	und als stark beschr\as nkt, falls
	\[\sup \set{\norm{x}_V \;\big\vert\; x\in A} < \infty\;.\]
	Wir zeigen hier tats\as chlich die \As quivalenz von starker und schwacher Beschr\as nktheit, die Gegenrichtung gilt ebenfalls. Die Konzepte stark (bzgl. \(\normn\cdot V\)) und schwach bzgl. aller \(\varphi\in V'\) lassen sich auch auf weitere Begriffe (z. B. Kompaktheit, Abgeschlossenheit) \us bertragen.
\end{rem}

\section{Der Satz von der offenen Abbildung}

\begin{rem}[Motivation]
	Seien \(V, W\) Banachr\as ume sowie \(T \in \B(V,W)\), d. h. \(T: V\to W\) bijektiv, linear und beschr\as nkt. Somit existiert \(T^{-1}:W\to V\). Die Umkehrabbildung ist linear, da aus
	\[T(T^{-1}(x+y)) = x +y \;\text{ und }\; T(T^{-1}(x) + T^{-1}(y)) = T(T^{-1}(x)) + T(T^{-1}(y)) = x + y\]
	unter Ausnutzung der Injektivit\as t von T folgt, dass \(T^{-1}(x+y) = T^{-1}(x) + T^{-1}(y)\). Weiterhin stellt sich die Frage, ob \(T^{-1}\) auch beschr\as nkt ist. Nach Satz \ref{B(V,W)_equiv} g. z. z., dass \(T^{-1}\) stetig. Daf\us r ist folgende Bedingung notwendig
	\[\forall G \subseteq V \text{ offen}: (T^{-1})^{-1} (G) = T(G) \subseteq W \text{ offen}\;.\]
\end{rem}
\begin{ex}
	Betrachte die Banachr\as ume \(V= W = (C^0([0,1]), \norms \cdot)\). Dann ist \(T\in \B(V,W)\)
	\[\forall f \in C^0([0,1]): T(f) = \int_0^x f(y)dy\]
	ein injektiver Operator. Dann k\os nnen wir \(\tilde T^{-1}\) definieren mit
	\[\tilde T^{-1} (g(x)) = g'(x)\;.\]
	Da \(T\) nicht bijektiv ist existiert in diesem Fall jedoch keine Umkehrabbildung.
\end{ex}

\begin{theorem}[Satz von der offenen Abbildung]
	Seien \(V, W\) Banachr\as ume und \(T\in \B(V,W)\). Falls \(T(V) = W\), dann ist f\us r alle \(G\subseteq V\) offen auch \(T(G)\) offen.
	\label{Offene_Abb} \end{theorem}
Wir zeigen zun\as chst zwei Lemmata.

\begin{lemma}
	Seien \(V,W\) Banachr\as ume und \(T\in \B(V,W),\;T(V) = W\). Dann gibt es ein \(d> 0\) mit 
	\[\forall \varepsilon > 0 \;\forall y\in W \;\exists x \in V: \norm{Tx-y} < \varepsilon \text{ und } \norm{x} < d^{-1}\norm{y}\;.\]
	\label{lemma1_Offene_Abb}
\end{lemma}
\begin{proof}
	Da T surjektiv ist, existiert f\us r \(y\in W\) beliebig ein \(\tilde x \in V\) mit \(T\tilde x = y\). Somit gilt
	\[W = \bigcup_{m\in \N} T(K_m(0))\;.\] 
	Da $W$ ein vollst\as ndiger metrischer Raum, k\os nnen wir die Kontraposition des Bairschen Kategorientheorems anwenden. Demnach existieren \(r\in \R_{>0},\;y_0 \in  W,\; m\in\N\) mit 
	\[K_r(y_0) \subseteq \overline{T(K_m(0))}\;.\]
	Falls f\us r \(\tilde y \in W: \norm{\tilde y}<r\), dann folgt \(y_0,\;y_0 + \tilde y \in K_r(y_0)\) und somit auch \(y_0,\;y_0 + \tilde y \in \overline{T(K_m(0))}\). Nach Definition des Abschluss existieren Folgen \(\seq{x^1}{n}, \;\seq{x^2}{n}\) in $V$ mit
	\[\forall n\in \N, \;i\in\set{1,2}: \norm{x^i_n}< m \;\text{ und }\; \lime{n}{T\left(x^1_n\right)} = y_0,\; \lime{n}{T\left(x^2_n\right)} = y_0 + \tilde y\;.\]
	Definiere nun die Folge \(\seq{x^3}{n}\) durch \(\forall n \in \N: x^3_n = x^2_n - x^1_n\). Damit gilt
	\[\forall n \in \N \norm{x^3_n} \leq 2m \;\text{ und }\; \lime{n}{T\left(x^3_n\right)} = \tilde y\;.\]
	Wir w\as hlen nun \(y\in W,\; y \neq 0\) (sonst \(x=0\)) und \(\varepsilon > 0\) fixiert. Wende nun obige Rechnung auf folgendes \(\tilde y\) an
	\[\tilde y = \frac{r}{2} \frac{y}{\norm y} \implies \norm{\tilde y} < r \implies \exists \seq{x^3}{n},\;\forall n\in \N\; x^3_n\in V,\; \norm{x^3_n} \leq 2m \text{ und }\lime{n}{T(x^3_n)} = \tilde y\;.\]
	Somit gilt 
	\[\lime{n}{\frac{2\norm y}{r} x^3_n} = y \implies \exists n\in \N: \norm{T\left(\frac{2\norm y}{r} x^3_n\right)-y}< \varepsilon\;.\]
	Setze nun \(d:=\frac{r}{4m}\), dann folgt
	\[\norm{\frac{2\norm y}{r} x^3_n} < \frac{2\norm y}{r} 2m = \frac{4m}r \norm{y} = d^{-1} \norm y\;.\]
	Dies war gerade zu zeigen.
\end{proof}

\begin{lemma}
	Seien \(V,W\) Banachr\as ume und \(T\in \B(V,W), T(V) = W\). Dann existiert ein \(\delta > 0\) mit 
	\[K_\delta(0) \subseteq T(K_1(0)) \iff \forall y\in W \text{ mit }\norm{y}<\delta \;\exists x\in V \text{ mit } \norm{x} < 1 \text{ und } Tx = y\;.\]
	\label{lemma2_Offene_Abb}
\end{lemma}
\begin{proof}
	Wir erhalten die Behauptung durch iteratives Anwenden des vorhergehenden Lemmas. Setzen \(\delta = d\), wobei $d$ wie im vorherigen Lemma sei. Sei weiterhin \(y_0:=y\in K_d(0)\) fixiert sowie \(\varepsilon_0 = \frac{d}{2}\). Dann liefert Lemma \ref{lemma1_Offene_Abb}
	\[\exists x_0 \in K_1(0): \norm{T(x_0) - y_0} < \frac{d}{2}\;.\]
	Dabei braucht nicht der Abschluss \(\overline{K_1(0)}\) zu betrachtet werden, weil nach dem Lemma \(\norm{x_0} \leq d^{-1} \norm{y_0} < d^{-1} d =1\). Erneutes Anwenden des Lemmas auf \(y_1:= y_0 -T(x_0)\) und \(\varepsilon_1 = \frac{d}{4}\) liefert
	\[\exists x_1 \in K_{\frac{1}{2}}(0): \norm{T(x_1) - y_1} = \norm{T(x_1) + T(x_0) - y} < \frac{d}{4}\]
	Dabei gilt \(x_1 \in K_{\frac{1}{2}}(0)\), da
	\[\norm{x_1} \leq d^{-1} \norm{y_1} = d^{-1} \norm{y_0 -T(x_0)} < \frac{d}{2d} = \frac{1}{2}\;.\]
	Iteration f\us hrt zu der allgemeinen Form, dass f\us r \(y_{n+1} = y_0 - T(x_n) -T(x_{n-1})- \ldots - T(x_0)\) gilt
	\[\exists x_{n+1} \in K_{2^{-(n+1)}}(0): \norm{T(x_{n+1})-y_{n+1}} =\norm{T(x_{n+1}) + T(x_n) + \ldots + T(x_0) - y} < d 2^{-(n+2)}\;.\]
	Nach Konstruktion gilt \(\sum_{n=0}^\infty \norm{x_n} < \sum_{n=0}^{\infty} 2^{-n} = 2 < \infty\). Somit ist \(\se{\sum_{n=0}^m x_n}{m}\) eine Cauchyfolge, denn es gilt f\us r \(m,m'\in \N, m' > m\)
	\[ \norm{\sum_{n=0}^{m'} x_n - \sum_{n=0}^{m} x_n} = \norm{\sum_{n=m+1}^{m'} x_n} \leq \sum_{n=m+1}^{m'} \norm{x_n} \overset{m\to\infty}{\longrightarrow} 0 \;.\]
	Da V ein Banachraum ist, existiert \(x = \lime{m}{\sum_{n=0}^m x_n}\), wobei \(\norm{x} < 2\). Da \(T\in \B(V,W)\) und somit insbesondere stetig, gilt auch \(\lime{m}{\sum_{n=0}^m T(x_n)} = T(x)\).
	Somit gilt insbesondere
	\[\lime{m}{y-\sum_{n=0}^m T(x_n)} = y - T(x)\;.\]
	Nach der obigen iterativen Konstruktion gilt aber auch
	\[\lime{m}\norm{y - \sum_{n=0}^m T(x_n)} = 0\;.\]
	Daraus folgt also 
	\[\forall y \in K_d(0) \;\exists x \in \overline{K_2(0)}: Tx = y\;.\]
	Setzen wir nun \(\delta = \frac{d}{2}\), folgt die Behauptung.	
\end{proof}

\begin{proof}[Beweis von Satz \ref{Offene_Abb}]
	Sei \(G\subseteq V\) offen und \(y\in T(G)\), d. h. es existiert \(x_0 \in G: y = T(x_0)\). F\us r Offenheit g. z. z., dass \(T(G)\) eine offene Kugel um \(y\) enth\as lt. Wir setzen
	\[\tilde G : = G -x_0 = \set{g-x_0\;\vert\;g\in G} \overset{T\text{ linear}}{\Longrightarrow} T(\tilde G) = T(G) - y\;.\]
 Somit g. z. z. \(\exists \tilde \delta > 0 : K_{\tilde \delta}(0)\subseteq T(\tilde G)\). Da \(\tilde G\) offen und \(0 \in \tilde G\), existiert \(\varepsilon > 0\) mit \(K_\varepsilon(0) \subseteq \tilde G\). Somit ergibt sich unter Ausnutzung der Linearit\as t von $T$ sowie mit  \(\delta\) wie in Lemma \ref{lemma2_Offene_Abb}
 \[T(\tilde G) \supseteq T(K_\varepsilon(0)) = \varepsilon T(K_1(0)) \supseteq \varepsilon K_\delta(0) = K_{\varepsilon\delta}(0)\;. \]
 Setzen also \(\tilde \delta = \varepsilon\delta\), dann gilt \(K_{\tilde \delta}(0) \subseteq T(\tilde G)\) \(\iff\) \(K_{\tilde \delta}(y) \subseteq T(G)\). Somit ist \(T(G)\) in der Tat offen.
\end{proof}
	
\begin{theorem}[Schlussfolgerung] Ist zus\as tzlich zu den Voraussetzungen von Satz \ref{Offene_Abb} $T$ injektiv (d. h. \(T\) bijektiv), dann gilt \(T^{-1} \in \B(W,V)\).
	\label{Inverse_Mapping}
\end{theorem}
\begin{proof}
	Wie bereits zuvor motiviert, g. z. z., dass \(T^{-1} \) beschr\as nkt und somit g. z. z., dass \(T^{-1}\) stetig. Es gilt 
	\[T^{-1}: W\to V \text{ stetig} \iff \forall G \subseteq V \text{ offen}: (T^{-1})^{-1}(G) = T(G) \subseteq W \text{ offen}\;.\] 
	Dies gilt nach Satz \ref{Offene_Abb} und somit ist \(T^{-1}\in \B(W,V)\).
\end{proof}



\section{Der Satz vom abgeschlossenen Graphen}
\begin{definition}
	Seien $X$, $Y$ normierte Vektorr\as ume, \(f:X \to Y\) eine lineare Abbildung. Dann bezeichnen wir 
	\[\Gamma_f = \set{(x,f(x)) \;\vert\; x \in X}\subseteq X\times Y\]
	als den \textit{Graph} von $f$.
\end{definition}

\begin{theorem}
	Seien \((V,\normn\cdot V)\), \((W,\normn\cdot W)\) Banachr\as ume. Dann wird \(V \times W\) unter eintragsweisen Operationen zu einem Vektorraum, d. h. 
	f\us r alle \( x,x_1\in V, y,y_1 \in W, \lambda \in \K\) gilt
	\[ \lambda(x, y) = (\lambda x, \lambda y) \;\text{ und }\;(x, y) + (x_1, y_1) = (x + x_1 , y+ y_1 )\;.\] 
	Mit jeder der Normen
	\[\normn{(x,y)}{1} := \normn xV + \normn yW ,\;\;\;\; \normn{(x,y)}{2} := \sqrt{\normn xV ^2 + \normn yW ^2},\;\;\;\; \normn{(x,y)}{3} := \max\set{\normn xV, \normn yW}\]
	wird \(V\times W\) ein Banachraum.
\end{theorem}
\begin{proof}
	Dass \(V\times W\) ein Vektorraum ist, ist aus der linearen Algebra bekannt. Zeige nun, dass \(\normn\cdot 1\) eine Norm ist. Es gilt \(\forall (x,y)\in V\times W\)
	\[0 = \normn{(x,y)}{1} \iff \normn xV + \normn yW = 0 \iff \normn xV = 0 \land \normn yW = 0 \iff x = 0 \land y = 0\;.\]
	Somit ist \(\normn\cdot 1\) definit. Die Homogenit\as t ist klar. Zeige nun die Dreiecksungleichung mit  
	\[\normn{(x + x_1, y+ y_1 )}{1} = \normn{x+ x_1}{V} + \normn{y+y_1}{W} \leq \normn xV + \normn{x_1}{V} + \normn yW + \normn{y_1}{W} = \normn{(x,y)}{1} + \normn{(x_1, y_1)}{1}\;.\]
	\(\normn\cdot 3\) ist offensichtlich eine Norm. F\us r \(\normn \cdot 2\) reicht es die Dreiecksungleichung zu zeigen, der Rest ist ebenfalls trivial. Es gilt
	\begin{align*} 
	\text{(i)}\;\normn{(x+x_1,y+y_1)}{2}^2& = \normn{x + x_1}{V}^2 + \normn{y + y_1}{W}^2 \leq (\normn xV + \normn{x_1}{V})^2 + (\normn yW  + \normn{y_1}{W})^2 \\
	&= \normn{x}{V}^2 + 2 \normn{x}{V}\normn{x_1}{V} + \normn{x_1}{V}^2 + \normn{y}{W}^2 + 2\normn yW \normn{y_1}{W} + \normn{y_1}{W}^2 \\
	\text{(ii)}\;(\normn{x,y}2 + \normn{x_1,y_1}2)^2 & = \left( \sqrt{\normn xV^2 + \normn yW^2} + \sqrt{\normn{x_1}V^2 + \normn{y_1}W^2}\right)^2 \\
	&= \normn xV^2 + \normn yW^2	 + \normn{x_1}V^2 + \normn{y_1}W^2 + 2 \sqrt{(\normn xV^2 + \normn yW^2)(\normn{x_1}V^2 + \normn{y_1}W^2)} \;.
	\end{align*}
	Somit gilt 
	\begin{align*}
	\text{(i)} \leq \text{(ii)} &\iff\normn{x}{V}\normn{x_1}{V} + \normn yW \normn{y_1}{W} \leq \sqrt{(\normn xV^2 + \normn yW^2)(\normn{x_1}V^2 + \normn{y_1}W^2)}  \\
	& \iff (\normn{x}{V}\normn{x_1}{V} + \normn yW \normn{y_1}{W})^2 \leq (\normn xV^2 + \normn yW^2)(\normn{x_1}V^2 + \normn{y_1}W^2) \\
	&\iff 2 \normn{x}{V}\normn{x_1}{V} \normn yW \normn{y_1}{W} \leq \normn xV^2 \normn{y_1}W^2 + \normn yW^2 \normn{x_1}V^2\\ 
	&\iff 0 \leq  (\normn xV \normn{y_1}W - \normn yW \normn{x_1}V)^2\;.
	\end{align*}
	Somit ist die Dreiecksungleichung in der Tat erf\us llt. Wir zeigen nun, Vollst\as ndigkeit bzgl. \(\normn\cdot 1\). Sei \(\se{(x_n, y_n)}{n},
	\; \forall n\in \N: x_n \in V, y_n \in W\) eine Cauchyfolge in \(V\times W\). Dann sind nach Definition auch \(\seq xn\) in $V$ und \(\seq yn\) in $W$ Cauchyfolgen, d. h. 
	\[\li xn := x \in V\text{ bzgl. } \normn\cdot V \;\;\text{ und }\;\;\li yn := y \in W\text{ bzgl. } \normn\cdot W \;\;\text{ existieren.}\]
	Damit w\as hlen wir \((x,y) \in V\times W\) als Kandidaten f\us r den Grenzwert von \(\se{(x_n, y_n)}n\). In der Tat
	\[\normn{(x_n, y_n) - (x,y)}1 = \normn{(x_n - x, y_n -y)}1 = \normn{x_n -x}V + \normn{y_n - y}W \overset{n\to\infty}{\longrightarrow} 0 \;.\]
\end{proof}
\begin{rem}
	Die Normen \(\normn\cdot 1,\;\normn\cdot 2,\; \normn\cdot 3\) auf \(V\times W\) sind 
	\as quivalent. Wir bezeichnen \(V\times W\) auch mit \(V\oplus W\).
\end{rem}

\begin{theorem}
	Seien \((H_1, \ip{\cdot, \cdot}_1)\) und \((H_2, \ip{\cdot, \cdot}_2)\) Hilbertr\as ume. 
	Dann ist \(H = H_1 \times H_2\) ein Hilbertraum mit dem Skalarprodukt
	\[\ip{(x_1,y_1), (x_2,y_2)} = \ip{x_1,x_2}_1 + \ip{y_1,y_2}_2\;.\]
\end{theorem}
\begin{proof}
	\happybegin
	Man kann leicht einsehen, dass \(\ip{\cdot,\cdot}\) die Norm \(\normn\cdot 2 \) induziert. Der Rest folgt trivial.\happyend
\end{proof}

\begin{theorem}[Satz vom abgeschlossenen Graphen]
Seien \((V,\normn\cdot V)\), \((W,\normn\cdot W)\) Banachr\as ume und \(T:V\to W\) linear. Dann gilt
\[T \in \B(V,W) \iff \Gamma_T \text{ abgeschlossen in }(V\times W)\;.\]
\label{closed_graph}
\end{theorem}
\begin{rem}
	\(\Gamma_T\) ist ein Untervektorraum von \(V\times W\), da unter Ausnutzung der Linearit\as t von $T$ gilt f\us r \(\lambda \in \K,\; x, x_1, x_2 \in V\)
	\begin{align*}
		& \lambda (x, Tx) = (\lambda x, \lambda Tx) = (\lambda x, T(\lambda x)) \\
		& (x_1, Tx_1) + (x_2, Tx_2) = (x_1 + x_2, Tx_1 + Tx_2) = (x_1 + x_2, T(x_1 + x_2))\;.
	\end{align*}
	Weiterhin ist \(\Gamma_T\) abgeschlossen, genau dann wenn 
	\[\forall \se{(x_n, y_n)}{n},\;\forall n\in\N \;(x_n, y_n) \in \Gamma_T \text{ und } \lime{n} (x_n, y_n) =:(x,y) \in V\times W \text{ existiert } \implies (x,y) \in \Gamma_T\;.\]
	Mit Satz \ref{B(V,W)_equiv} l\as sst sich dies \as quivalent umformulieren zu
	\[\forall \seq xn \text{ mit }\;\forall n\in \N\; x_n \in V,\;\li xn =: x \in V  \text{ und } \lime n {Tx_n} = y \in W\text{ existieren } \implies y = T(x)\]
\end{rem}
\begin{proof}
	\happybegin
	\forw Sei \(T\in \B(V,W)\) und sei \(\se{(x_n, Tx_n)}{n}, \forall n\in \N\; x_n \in V\) eine Folge in \(\Gamma_T\), mit \(\lime n (x_n, Tx_n) = (x,y) \in V\times W\).  Somit 
	\[\lime n\normn{x_n - x}{V}  = 0 \text{ und } \lime n \normn{Tx_n - y}{W} = 0\;.\]
	Aufgrund der Stetigkeit von $T$ folgt in der Tat \(T(x) = y\).\\
	\backw Sei \(\Gamma_T\) abgeschlossen, dann ist \(\Gamma_T\) als Unterraum des Banachraums \((V\times W,\normn\cdot 1)\) ebenfalls ein Banachraum. Wir definieren die stetigen, linearen Abbildungen
	\[P_1: \Gamma_T \to V,\; P_1(x,Tx) = x \;\;\text{ und }\;\; P_2: \Gamma_T\to W,\; P_2(x,Tx) = Tx\;.\]
	Dabei gilt f\us r die Operatornorm $\norm{P_1}$
	\begin{multline*}
	\normn{P_1(x,Tx)}V = \normn xV \leq \normn xV + \normn{Tx}{W} = \normn{(x,Tx)}{1} \implies \\ \norm{P_1} = \sup \set{ \normn{P_1(x,Tx)}V\;\big\vert\;\normn{(x,Tx)}1 = 1}\leq 1\;.
	\end{multline*}
	Analog folgt \(\norm{P_2} \leq 1\). Weiterhin gilt \(P_1(\Gamma_T) = V\), d. h. $P_1$ injektiv. Au\s erdem ist $P_1$ injektiv, da
	\[P_1(x,Tx) = 0 \iff x = 0 \iff x = 0 \text{ und } Tx = 0 \iff (x,Tx) = 0_{V\times W}\;.\]
	Aus der Bijektivit\as t von $P_1$ folgt nun mit Satz \ref{Inverse_Mapping} \(P_1^{-1} \in \B(V,\Gamma_T)\) und mit \(P_2 \in \B(\Gamma_T, W)\) nach Definition erhalten wir 
	\[T = P_2 \circ P_1^{-1} \in \B(V,W)\;.\]
	Dies war  gerade zu zeigen \happyend.
\end{proof}

\begin{theorem}
	Sei $V$ ein Vektorraum mit zwei Normen \(\normn\cdot 1, \;\normn\cdot 2: V \to \R_{\geq 0}\), wobei \((V,\normn\cdot 1)\) und \((V,\normn\cdot 2)\) Banachr\as ume sind. Au\s erdem gilt 
	\begin{multline*}
	\forall \seq xn,\; \forall n \in \N\; x_n \in V:\big( \li xn := x\in V \text{ bzgl. }\normn\cdot 1 \text{ und } \\\li xn := y\in V \text{ bzgl. }\normn\cdot 2 \text{ existieren }\big)  \implies x = y\;.
	\end{multline*}
	Dann sind \(\normn\cdot 1\) und \(\normn\cdot 2\) \as quivalent.
\end{theorem}

\begin{proof}
	Definiere \(I: (V,\normn\cdot1) \to (V,\normn\cdot 2)\) als die Identit\as tabbildung, welche linear ist. Wollen nun zeigen, dass \(\Gamma_T\) abgeschlossen. Sei daf\us r eine Folge \(\seq xn, \;\forall n \in \N\; x_n \in V\) gegeben mit \(\li xn =: x\) in \((V,\normn\cdot 1)\) sowie \(\lime n Tx_n = y\) in \(V,\normn\cdot 2\). Nach Voraussetzung folgt \(x = y\). Somit ist \(\Gamma_I\) abgeschlossen und mit Satz \ref{closed_graph} gilt \(I \in \B(V,V)\). Nach Satz \ref{Inverse_Mapping} gilt auch \(I^{-1} \in \B(V,V)\). Nach Definition der Beschr\as nktheit folgt die Behauptung.
\end{proof}