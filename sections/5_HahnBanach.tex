\section{Der Satz von Hahn-Banach}
\begin{definition}
	Eine partielle Ordnung auf einer Menge $X$ ist eine Relation $\leq$, sodass f\us r alle \(a,b,c \in X\) gilt:
	\begin{enumerate}[noitemsep]
		\item Transitivit\as t: \(a\leq b \land b \leq c \implies a\leq c\)
		\item Reflexivit\as t: \(a\leq a\)
		\item Anti-Symmetrie: \(a\leq b \land b\leq a \implies a = b\)	
	\end{enumerate}
	Falls f\us r alle \(a,b \in X\) gilt \(a\leq b \lor b \leq a\), dann wird $X$ als total geordnet bezeichnet.
\end{definition}
\begin{theorem}[Lemma von Zorn]
	Sei \((Z, \leq)\) eine partiell geordnete Menge mit der Eigenschaft, dass jede total geordnete Teilmenge (bez. \textit{Kette}) eine obere Schranke in $Z$ hat. Dann hat $Z$ ein maximales Element. 
	\end{theorem}
	
	\begin{lemma}
		Sei \((V,\norm\cdot)\) ein normierter Vektorraum, \(U\subseteq V\) ein Unterraum, \(z \in V\setminus U\) so, dass \(V = \text{span}\set{U \cup \set{z}}\) sowie \(\varphi_0 \in U'\). Dann gibt es \(\varphi \in V'\) mit \(\forall u \in U: \varphi(u) = \varphi_0(u)\) und \(\norm{\varphi} = \norm{\varphi_0}\).
		\unsure{In der Vorlesung wurde im Lemma \(\norm{\varphi} \geq \norm{\varphi_0}\) angegeben, aber dies sollte direkt aus \(\varphi\vert_U = \varphi_0\) folgen und wir zeigen im Lemma \(\leq\) und somit Gleichheit, oder?}
		\label{Lemma_Hahn_Banach}
	\end{lemma}
	\begin{proof}
		Wir zeigen hier nur den Fall \(\K = \R\). Nehmen o. B. d. A. an, dass \(\varphi_0 \neq 0\) (ansonsten setze \(\varphi = 0\)) und \(\norm{\varphi_0} = 1\). Weiterhin gilt nach Voraussetzung
		\[V = \text{span}\set{U\cup\set{z}} = \set{u + \alpha z \;\vert\; u \in U, \alpha \in \R} = \set{\alpha z - u \;\vert\; u \in U, \alpha\in \R}\;.\]
		Zeige nun, dass die Darstellung \(v = \alpha z - u \in V\) durch \(u\in U, \alpha \in \R\) eindeutig ist. Angenommen f\us r \(u_1, u_2 \in U, \alpha_1, \alpha_2 \in \R\) gilt \(v = u_1 + \alpha_1 z = u_2 + \alpha_2 z\), dann folgt
		\[u_1 + \alpha_1 z = u_2 + \alpha_2 z \iff \overbrace{(\alpha_2 - \alpha_1)z}^{\not\in U} = \overbrace{u_1 - u_2}^{\in U}\implies \alpha_1 = \alpha_2 \implies u_1 = u_2\;.\]
		Betrachte nun \(x,y \in U\), dann gilt (mit \(\norm{\varphi_0} = 1\))
		\begin{align*}&\overbrace{\varphi_0(x) -\varphi_0(y)}^{\in \R} \leq \ab{\varphi_0(x) -\varphi_0(y)} = \ab{\varphi_0(x) -\varphi_0(y)} \\= &\ab{\varphi_0(x-y)} \leq \norm{\varphi_0} \norm{x-y}  
			= \norm{x-y} \leq \norm{x-z} + \norm{y-z} \\ \iff &\varphi(x) - \norm{x-z} \leq \varphi_0(y) + \norm{y-z}\;.
			\end{align*}
			Somit existiert \(c\in \R\) mit 
			\[\sup_{x\in U}\varphi_0(x) - \norm{x-z}\leq c \leq \inf_{y\in U} \varphi_0(y) + \norm{y-z}\;.\]
			Damit k\os nnen wir nun die Forsetzung definieren, wobei f\us r \(v \in V\) (wie oben) gilt
			\[\varphi(v) = \varphi(\alpha z - u) := \alpha c - \varphi_0(u)\;.\]
			Wir pr\us fen nun die geforderten Eigenschaften. Es gilt offensichtlich \(\varphi\vert_U = \varphi_0\) (setzen \(\alpha = 0\) in eindeutiger Darstellung) und somit automatisch 
			\(\norm{\varphi} \geq \norm{\varphi_0}\). Zeige nun die Normabsch\as tzung. Zun\as chst gilt  \(\forall\tilde u \in U\)
			\begin{align*}
			& \varphi_0(\tilde u) - \norm{\tilde u - z} \leq c \leq \varphi_0 (\tilde u) + \norm{\tilde u -z} \\
			\iff & 	- \norm{\tilde u - z} \leq c - \varphi_0 (\tilde u) \leq \norm{\tilde u -z}	
			\\
			\iff & \ab{c-\varphi_0(\tilde u)} \leq \norm{\tilde u -z}
			\end{align*}
			Somit erhalten wir \(\forall \alpha \in \R, \alpha \neq 0, u \in U\) (d.h. f\us r \(v\in V\setminus U\))
			\[\ab{\varphi(\alpha z -u)} = \ab{\alpha} \ab{\varphi\left(z - \frac{u}{\alpha}\right)} = \ab{\alpha} \cdot \ab{c - \varphi_0\left(\frac{u}{\alpha}\right)} \leq \ab{\alpha}\norm{\frac{u}{\alpha}-z} = \norm{\alpha z -u}\;.\]
			Folglich gilt 
			\(\ab{\varphi(\alpha z -u)} \leq \norm{\varphi_0} \norm{az - u}\implies \norm{\varphi} \leq \norm{\varphi_0}\) und es folgt Normgleichheit.
	\end{proof}
	
\begin{theorem}[Hahn-Banach]
	Sei \((V,\norm\cdot)\) ein normierter Vektorraum, \(U\subseteq V\) ein Unterraum und \(\varphi_0 \in U'\). Dann gibt es ein \(\varphi \in V'\) mit \(\varphi\vert_U = \varphi_0\) und \(\normn{\varphi}{V'} = \normn{\varphi_0}{U'}\). D. h. jedes beschr\as nkte, lineare Funktional kann normgleich fortgesetzt werden. 
\end{theorem}
\begin{proof}
	O. B. d. A \(\norm{\varphi_0} = 1\). Wir setzen
	\[Z := \set{(W,\psi)\;\big\vert\; U\subseteq W \subseteq V \text{ Unterraum, }\;\psi \in W',\; \psi\vert_U = \psi_0,\; \norm{\psi} = \norm{\varphi_0} = 1}\;.\]
	Definieren partielle Ordnung auf $Z$ \us ber
	\[(W,\psi) \leq (\tilde W, \tilde \psi) : \iff W \subseteq \tilde W \text{ und } \tilde\psi \vert_W = \psi\;.\]
	Sei nun $Z_0\subseteq Z$ eine total geordnete Teilmenge, womit wir $W^*\subseteq V$ wie folgt definieren
	\[Z_0 := \set{(W_i,\psi_i)\;\vert\; i \in I}\;\text{ und }\; W^* := \bigcup_{i\in I} W_i\;.\]
	Dabei ist $W^*$ ein Unterraum, da f\us r \(x,y \in W^*\) existieren \(i_1, i_2 \in I\), sodass \(x \in W_{i_1}\) und \(y\in W_{i_2}\). Sei o. B. d. A. \((W_{i_2}, \psi_{i_2}) \geq (W_{i_1}, \psi_{i_1})\), dann gilt nach Voraussetzung auch \(x\in W_{i_2}\), dies ist ein Unterraum und somit \(x+y \in W^*\). Wir definieren nun ein lineares Funktional \(\psi^*\) auf \(W^*\), wobei f\us r \(w \in W^*\)
	\[ \exists \hat i\in I: w \in W_{\hat i} \text{ und somit }\psi^*(w) := \psi_{\hat i}(w)\;.\]
	Zeige nun, dass \(\psi^*\) wohldefiniert ist. Seien \((W_{i_j}, \psi_{i_j})\in Z_0\), \(j\in\set{1,2}\), \(w\in W_{i_1}\), o. B. d. A. \(W_{i_1}\subseteq W_{i_2}\) und \(\psi_{i_2}\vert_{W_{i_1}} = \psi_{i_1}\). Dann gilt \(\psi_{i_2}(w) = \psi_{i_1}(w)\), d. h. 
	\(\psi^*\) ist wohldefiniert. Weiterhin existiert f\us r alle \(w\in W^*\) ein \(i^* \in I \) mit 
	\[\ab{\psi^*(w)} = \ab{\psi_{i^*}(w)} \leq \norm{\psi_{i^*}}\norm{w} = \norm{\varphi_0} \norm{w} = \norm{w} \implies \norm{\psi^*} \leq 1\;.\]
	Da offensichtlich \(\norm{\psi^*} \geq \norm{\varphi_0}\) (wegen \(\psi^*\vert_U = \varphi_0\)), folgt \(\norm{\psi^*} = 1\).
	Somit \((W^*, \psi^*)\in Z\) und f\us r \((W,\psi) \in Z_0\) beliebig gilt nach Konstruktion 
	\[W \subseteq W^* \text{ und } \forall w \in W: \psi^*(w) = \psi(w) \implies (W,\psi) \leq (W^*, \psi^*)\;.\]
	Somit ist \((W^*, \psi^*)\) eine obere Schranke von $Z_0$. Nach dem Lemma von Zorn hat $Z_0$ also ein Maximum, bez. dieses mit \((W_0,\psi_0)\). Angenommen \(W_0 \neq V\). Dann k\os nnen wir f\us r \(z \in V \setminus W_0\) Lemma \ref{Lemma_Hahn_Banach} anwenden auf \(W_1:= \text{span}\set{W_0\cup\set{z}}\), d. h.
	\[\exists \eta \in W_1' :\; \eta\vert_{W_0} = \psi_0 \text{ und } \norm{\eta} = \norm{\psi_0} = \norm{\varphi_0}\;.\]
	Dann folgt jedoch \((W_1,\eta) \in Z\) mit \((W_1,\eta) > (W_0,\psi_0)\), was ein Widerspruch dazu ist, dass \((W_0, \psi_0)\) eine obere Schranke ist. Somit leistet \(W_0 = V\) und \(\varphi:= \psi_0\) das Verlangte.
\end{proof}
	
	\begin{ex}
		Sei $V$ ein $\K$-Vektorraum, \(U\subseteq V\) ein dichter Teilraum von $V$ und \(\varphi_0\in U'\). Nach Voraussetzung existiert f\us r jedes \(x\in V\) eine Cauchyfolge  \(\seq xn\), \(\forall n\in \N: x_n \in U\) mit \(\li xn = x\). Da $\varphi_0$ beschr\as nkt und linear (und somit stetig), folgt, dass auch \(\se{\varphi_0(x_n)}{n}\) eine Cauchyfolge auf  \(\K = \R\) oder \(\K = \C\) (und somit konvergent) ist. Wir definieren die (eindeutige) Fortsetzung \(\varphi \in V'\) mit
		\[\forall x\in V' : \varphi(x) = \lim_{n\to\infty} \varphi_0(x_n)\;.\]
		Weiterhin gilt
		\[\lime n \varphi_0(x_n) \leq \lime n\norm{\varphi_0} \norm{x_n} = \norm{\varphi_0}\norm{x}\;. \]
		\unsure{Es ist nicht ganz klar, wozu die letzte Absch\as tzung n\os tig ist und wieso die Eindeutigkeit gilt.}
	\end{ex} 
	
	\begin{theorem}
		Sei \((V,\norm\cdot)\) ein normierter Raum, wobei \(V \neq \set{0}\). Dann gibt es f\us r jedes \(x_0\neq 0 \in V\) ein \(\varphi \in V'\) mit \(\varphi(x_0) = \norm{x_0}\) und \(\norm{\varphi} = 1\).
	\end{theorem}
	\begin{proof}
		Wir setzen \(U:=\set{\alpha x_0\;\vert\; \alpha \in \K}\), damit ist \(U\subseteq V\) ein Unterraum. Definieren \(\varphi_0\in U'\) \us ber 
		\[\forall \alpha \in \K: \varphi_0(\alpha x_0) = \alpha \norm{x_0} \implies \varphi_0(x_0) = \norm{x_0}\;.\]
		Weiterhin gilt
		\[\norm{\varphi_0} = \sup \set{\ab{\varphi_0(\alpha x_0)} \;\big\vert\; \alpha \in \K, \ab{\alpha}\norm{x_0} \leq 1} = \sup \set{\ab{\alpha} \norm{x_0} \;\big\vert\; \ab{\alpha}\norm{x_0} \leq 1} = 1\;.\]
		Nach dem Satz von Hahn-Banach k\os nnen wir nun \(\varphi_0\) normgleich nach \(V'\) fortsetzen und erhalten somit die geforderten Eigenschaften.
	\end{proof}	
	
	\begin{theorem}
		\label{Doppeldual_1} Sei \((V,\norm\cdot)\) ein normierter Raum. Dann ist 
		\[\psi: V \to V'',\; x \mapsto (x'':\varphi \mapsto \varphi(x))\]
		eine isometrische Abbildung.
	\end{theorem}
	\begin{proof}
		Es gilt 
		\[\norm{x''} = \sup\set{\ab{\varphi(x)} \;\big\vert\;\varphi\in V', \norm{\varphi} \leq 1} \]
		Dabei gilt \(\ab{\varphi(x)} \leq \norm{\varphi}\norm{x} \leq \norm{x}\) f\us r \(\norm{\varphi}\leq 1\). Somit \(\norm{x''} \leq \norm{x}\). Da \(\varphi_x \in V'\) mit \(\varphi_x(x):= \norm{x}\), folgt die Gleichheit \(\norm{x''} = \norm{x}\).
			
	\end{proof}
	
