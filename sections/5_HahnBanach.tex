\section{Der Satz von Hahn-Banach}
\begin{definition}
	Eine partielle Ordnung auf einer Menge $X$ ist eine Relation $\leq$, sodass f\us r alle \(a,b,c \in X\) gilt:
	\begin{enumerate}[noitemsep]
		\item Transitivit\as t: \(a\leq b \land b \leq c \implies a\leq c\)
		\item Reflexivit\as t: \(a\leq a\)
		\item Anti-Symmetrie: \(a\leq b \land b\leq a \implies a = b\)	
	\end{enumerate}
	Falls f\us r alle \(a,b \in X\) gilt \(a\leq b \lor b \leq a\), dann wird $X$ als total geordnet bezeichnet.
\end{definition}
\begin{theorem}[Lemma von Zorn]
	Sei \((Z, \leq)\) eine partiell geordnete Menge mit der Eigenschaft, dass jede total geordnete Teilmenge (bez. \textit{Kette}) eine obere Schranke in $Z$ hat. Dann hat $Z$ ein maximales Element. 
	\end{theorem}
	
	\begin{lemma}
		Sei \((V,\norm\cdot)\) ein normierter Vektorraum, \(U\subseteq V\) ein Unterraum, \(z \in V\setminus U\) so, dass \(V = \text{span}\set{U \cup \set{z}}\) sowie \(\varphi_0 \in U'\). Dann gibt es \(\varphi \in V'\) mit \(\forall u \in U: \varphi(u) = \varphi_0(u)\) und \(\norm{\varphi} \geq \norm{\varphi_0}\).
	\end{lemma}
	\begin{proof}
		Wir zeigen hier nur den Fall \(\K = \R\). Nehmen o. B. d. A. an, dass \(\varphi_0 \neq 0\) und \(\norm{\varphi_0} = 1\). Weiterhin gilt nach Voraussetzung
		\[V = \text{span}\set{U\cup\set{z}} = \set{u + \alpha z \;\vert\; u \in U, \alpha \in \R} = \set{\alpha z - u \;\vert\; u \in U, \alpha\in \R}\;.\]
		Zeige nun, dass f\us r \(v = \alpha z - u \in V\) mit \(u\in U, \alpha \in \R\) eindeutig ist. Angenommen f\us r \(u_1, u_2 \in U, \alpha_1, \alpha_2 \in \R\) gilt \(v = u_1 + \alpha_1 z = u_2 + \alpha_2 z\), dann folgt
		\[u_1 + \alpha_1 z = u_2 + \alpha_2 z \iff \overbrace{(\alpha_2 - \alpha_1)z}^{\not\in U} = \overbrace{u_1 - u_2}^{\in U}\implies \alpha_1 = \alpha_2 \implies u_1 = u_2\;.\]
		Betrachte nun \(x,y \in U\), dann gilt (mit \(\norm{\varphi_0} = 1\))
		\begin{align*}&\overbrace{\varphi_0(x) -\varphi_0(y)}^{\in \R} \leq \ab{\varphi_0(x) -\varphi_0(y)} = \ab{\varphi_0(x) -\varphi_0(y)} \\= &\ab{\varphi_0(x-y)} \leq \norm{\varphi_0} \norm{x-y}  
			= \norm{x-y} \leq \norm{x-z} + \norm{y-z} \\ \iff &\varphi(x) - \norm{x-z} \leq \varphi_0(y) + \norm{y-z}\;.
			\end{align*}
			Somit existiert \(c\in \R\) mit 
			\[\sup_{x\in U}\varphi_0(x) - \norm{x-z}\leq c \leq \inf_{y\in U} \varphi_0(y) + \norm{y-z}\;.\]
			Damit k\os nnen wir nun die Forsetzung definieren, wobei f\us r \(v \in V\) (wie oben) gilt
			\[\varphi(v) = \varphi(\alpha z - u) := \alpha c - \varphi_0(u)\;.\]
			Wir pr\us fen nun die geforderten Eigenschaften. Es gilt offensichtlich \(\varphi\vert_U = \varphi_0\) (setzen \(\alpha = 0\) in eindeutiger Darstellung). Zeige nun die Normabsch\as tzung. Zun\as chst gilt  \(\forall\tilde u \in U\)
			\begin{align*}
			& \varphi_0(\tilde u) - \norm{\tilde u - z} \leq c \leq \varphi_0 (\tilde u) + \norm{\tilde u -z} \\
			\iff & 	- \norm{\tilde u - z} \leq c - \varphi_0 (\tilde u) \leq \norm{\tilde u -z}	
			\\
			\iff & \ab{c-\varphi_0(\tilde u)} \leq \norm{\tilde u -z}
			\end{align*}
			Somit erhalten wir \(\forall \alpha \in \R, \alpha \neq 0, u \in U\) (d.h. f\us r \(v\in V\setminus U\))
			\[\ab{\varphi(\alpha z -u)} = \ab{\alpha} \ab{\varphi\left(z - \frac{u}{\alpha}\right)} = \ab{\alpha} \cdot \ab{c - \varphi_0\left(\frac{u}{\alpha}\right)} \leq \ab{\alpha}\norm{\frac{u}{\alpha}-z} = \norm{\alpha z -u}\;.\]
			[...\textit{Was sagen die letzten beiden Schritte?, folgt Normabsch\as tzung nicht schon aus \(\varphi\vert_U = \varphi_0\)?}]
	\end{proof}
	
\begin{theorem}[Hahn-Banach]
	Sei \(V,\norm\cdot\) ein normierter Vektorraum, \(U\leq V\) ein Unterraum und \(\varphi_0 \in U'\). Dann gibt es ein \(\varphi \in V'\) mit \(\varphi\vert_U = \varphi_0\) und \(\normn{\varphi}{V'} = \normn{\varphi_0}{U'}\). D. h. jedes beschr\as nkte, lineare Funktional kann normgleich fortgesetzt werden. 
\end{theorem}