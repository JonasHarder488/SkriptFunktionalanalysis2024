
\begin{definition} $S, T \in \mc B(H)$ selbstadjungiert, dann $S \leq T \iff T - S \geq 0.$

\begin{rem} Vorsicht: $\text{Re} \B H$ ist kein Verband, d.h: \[\exists S, T \geq 0  \not \exists ! U \geq 0 \  \forall V \geq S, V \geq T: V \geq U\]
\end{rem}

\end{definition}


\begin{theorem} Sei $S, T, U \in \mc B(H)_{\geq 0}, \lambda \geq 0$

\begin{enumerate}
	
	\item $S + T \geq 0$
	
	\item $\lambda S \geq 0$
	
	\item $ST = TS \implies ST \geq 0.$
	
\end{enumerate}


\begin{proof}[Beweis:] Folgt aus dem Spektralsatz für kommutative Operatoren.
	
\end{proof}

\end{theorem}


\begin{theorem} Sei $T \in \mc B(H),$ dann ist $\norm{T} = \sqrt{\inf{\set{\lambda \geq 0: \lambda \text{Id} - T^*T \geq 0}}}$

\begin{proof}[Beweis:] Wir wissen $\norm{T}^2 = \norm{T^*T}$ und zeigen nun für $S \geq 0,$ dann \[\norm{S} = \inf \set{\lambda \geq 0: \lambda \text{If}} \geq S =: \norm{S}'.\]
	
Für alle $\lambda \geq \norm{S}$ und $\norm{x} = 1$ gilt: \[\lambda = \lambda \norm{x}^2 \geq \ip{x, Sx} \implies \lambda \geq \norm{S} \implies \norm{S} \leq \norm{S}'\] \\
	
Sei nun $\lambda \geq \norm{S},$ dann für alle $x \in H, \norm{x} = 1:$
	
\[\ip{x, Sx} \leq \norm{Sx} \leq \norm{S}\norm{x} = \norm{S}\] Damit haben wir $\norm{S} = \norm{S}^*.$
	
\end{proof}
	
\end{theorem}


\begin{theorem} Seien $A_n \in \mc B(H)_{\geq 0}, A \in \mc B(H)$ und für alle $n \in \N$ gelte $A \geq A_{n+1} \geq A_n,$ dann existiert $\sup_{n\in \N}\set{A_n} =: B$ und für alle $x, y \in H$ gilt: \[\ip{x, A_n y} \xrightarrow{n \to \infty} \ip{x, By}.\]
	
	
	\begin{proof}[Beweis:] Wir setzen für $x, y \in H: L_B = \ip{ x, A_ny}.$ Wir observieren: \[\forall x \in H: \lim_{n \to \infty} \ip{x, A_nx} = \sup_{n \in \N}{\ip{x, A_nx}} \implies L_B(x,x) \text{ ist definiert.}\] Betrachte nun $x, y \in H.$ Wir wenden die Polarisationsformel an: \[\lim_{n \to \infty} \ip{x, A_n y} = \lim_{n \to \infty} \frac{1}{4} \sum^n_{n = 0}(-1)^k \ip{x+iy^k, A_n x + i^k y} \implies L_B(x, y) \text{ existiert.}\] Außerdem gilt \[L_B(x+y, x+y) + L_B(x - y, x-y) = 2L_B(x, x) + 2L_B(y, y).\] Somit ist $L_B$ eine Sesquilinearform. Da wir nun wissen, dass $A_n \leq A,$ folgt für alle \[x \in H: L_B(x, x) = \lim_{n \to \infty} \ip{x, A_nx} \leq \ip{x, Ax} \leq \norm{A} \norm{x}^2\] Und über die Sesquilinearität folgt die Beschränktheit, mit Riez die Eindeutigkeit.
		
		\[\forall x, y \in H: \norm{L_B(x, y) \leq 4 \norm{A} \norm{x} \norm{y}} \implies \exists ! B \in \mc B(H): L_B(x, y) = \ip{x, By}\]
		
	\end{proof}
	
\end{theorem}


\begin{definition} Ein Operator $T \in \mc B(H)$ gehört zur Spurklasse genau dann wenn es eine ONB $(e_s)_{s \in S}$ gibt mit der Spur $\text{tr}(T) = \sum_{s \in S} \ip{e_s, Te_s} < \infty.$
	
\end{definition}


\begin{theorem} $T \geq 0$ gehört zur Spurklasse genau dann wenn $\sqrt{\ab{T}}$ ein Hilbert-Schmidt-Operator ist. Es gilt dann für alle ONB‘s: \[\sum_{s \in S} \ip{e_s, Te_s} = \sum_{s \in S} \ip{e_s', Te_s'}\]
	
	
	\begin{proof}[Beweis:] $(\Rightarrow)$ Gehöre $T$ zu Spurklasse:
		
	\[ \infty > \sum_{s \in S} \ip{e_s, \ab{T} e_s} = \sum_{s \in S} \ip{\sqrt{\ab{T}e_s, \sqrt{\ab{T}}e_s}} = \sum_{s \in S} \norm{\sqrt{\ab{T}}e_s}^2\] Damit ist $\sqrt{\ab{T}}$  ein Hilbert Schmidt Operator
			
	$(\Leftarrow)$ Sei $T \geq 0,$ dann ist $T$ selbstadjungiert. Sei nun \[T_+ = \frac{1}{2}(T + \ab{T}) \geq 0, T_- = \frac{1}{2}(\ab{T}- T) \geq 0, T = T_+ - T_-.\] Dann haben wir \[\sum_{s \in S} \ip{e_s, T_+e_s} = \sum_{s \in S'} \ip{e_s, T_+ e_s} \implies (*).\] Sei $T \in \mc B(H)$ beliebig, dann $\text{Re}(T) = \frac{1}{2} (T + T^*), \text{Im}(T) = \frac{1}{2}(T-T^*)$ sind selbstadjungiert mit $T = \text{Re}(T)= + \text{Im}(T).$ 
	\end{proof}
\end{theorem}

\begin{theorem}
	Jeder Spurklasseoperator auf einem Hilbertraum $H$ ist kompakt. Weiterhin gilt f\us r $T$ Spurklasse sowie \(S \in B(H)\) auch \(TS, ST\) Spurklasse und \(\tr (ST) = \tr(TS)\).
\end{theorem}
\begin{proof}[Beweis]
	Sei also \(S\in \B(H)\) und $T$ Spurklassse.  Da die Isometrie nach Satz \ref{stetiger_kalkuel} involutorisch ist, gilt
	\[\ab{ST} = \sqrt{(ST)^*ST} = \sqrt{T^*S^*ST}\;.\]
	Weiterhin gilt mit Satz \unsure{Label fehlt (Satz 8.19 in Vorlesung)}
	\[S^*S = \ab{S}^2  \leq \norm{S}^2 \I \implies T^* S^* ST \leq T^*T\norm S^2 \;.\]
	Somit folgt
	\[\sqrt{T^*S^*ST} \leq \norm{S} \sqrt{T^*T} = \norm S \ab T\;.\]	
	Dies ergibt f\us r eine ONB \((e_s)_{s\in S}\) von $H$
	\[\sum_{s\in S}\ip{e_s, \ab{ST} e_s} \leq \sum_{s\in S} \norm{S} \ip{e_s, \ab{T}e_s} < \infty\;.\]
	Die Behauptung f\us r $ST$ erhalten wir \us ber \(TS = (S^*T^*)^*\). Weiterhin gilt
	\begin{align*}\tr(ST) &= \sum_{s\in S} \ip{e_s, ST e_s} = \sum_{s\in S} \ip {S^* e_s, T e_s} = \sum_{s, s' \in S}\ip{S^* e_s, e_{s'}}\ip{e_{s'}, Te_s} \\
		&= \sum_{s, s' \in S}\ip{T^* e_{s'}, e_s}\ip{e_s, Se_{s'}} = \sum_{s'\in S} \ip{T^* e_{s'}, S e_{s'}} = \sum_{s'\in S} \ip {e_{s'}, TS e_{s'}}\;.
		\end{align*}
\end{proof}
\begin{theorem}
	Sei \(T\geq 0\) Spurklasse auf einem Hilbertraum $H$. Dann existiert eine Folge \(\seq \lambda n \subseteq \R^\N\) mit \( \forall n\in\N: \lambda_n \geq 0, \sum_{n\in\N} \lambda_n <\infty\) sowie ein ONS \(\seq en \subseteq H^\N\) sodass gilt
	\[ T = \sum_{n\in\N} \lambda_n|e_n\rangle \langle e_n|\;.\]
	\label{spurklasse_geq0_Folge}
\end{theorem}
\begin{proof}[Beweis]
	Die Behauptung folgt aus \(\sum_{n\in\N} \lambda_n = \tr(T)\).
\end{proof}