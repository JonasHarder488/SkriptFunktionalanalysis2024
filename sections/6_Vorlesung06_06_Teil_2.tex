\textit{[*****Hier fehlt noch der 1. Teil der Vorlesung vom 06.06.*****]}
\begin{theorem}
	Seien \(V,W, U, X\) Banachr\as ume und  \(S_1\in B(W,X)\), \(S_2\in B(U,V)\), \;\(\alpha \in \K\) sowie \(T_1, T_2\in \mc K(V,W)\). Dann sind \(T_1 + T_2, \alpha T_1 \in \mc K(V,W)\) und \(S_1 T_1\in \mc K(V,X)\), \(T_1S_2\in \mc K(U,W)\). D. h. die Addition kompakter Operatoren und deren Multiplikation mit Skalaren ergibt wieder einen kompakten Operator. Weiterhin ist die beidseitige Komposition eines beschr\as nkten mit einem kompakten Operator wieder kompakt.
\end{theorem}
\begin{proof}
	Betrachte zun\as chst die Addition \(T_1 + T_2\). Sei \(\seq xn,\; \forall n\in\N: x_n \in V\) eine beschr\as nkte Folge. Da $T_1$ kompakt, gilt nach Definition
	\[\exists \text{ TF } (x_{n_k})_{k\in\N}: (T_1 x_{n_k})_{k\in\N} \text{ konvergiert in } W\;.\]
	Da die Folge \(\se {x_{nk}}k\) ebenfalls wieder beschr\as nkt ist und $T_2$ kompakt, gilt
	\[\exists\text{ TTF } \se {x_{n_{k_l}}}l:  \se { T_2 x_{n_{k_l}}}l \text{ und } \se {T_ 1x_{n_{k_l}}}l \text{ konvergieren in }W \;.\]
	Damit konvergiert aber auch die Folge
	\[\se {(T_1 + T_2) x_{n_{k_l}}}l = \se { T_2 x_{n_{k_l}}}l + \se {T_ 1x_{n_{k_l}}}l \]
	in W. Somit ist \(T_1 + T_2 \in \mc K(V,W)\). \(\alpha T_1 \in \mc K(V,W)\) ist trivial. Betrachte nun \(S_1 T_1\). Nach Satz \textit{[...]} \unsure{Label fehlt} gen\us gt zu zeigen, dass \(S_1 T_1 \overline{K_1^V(0)}\) totalbeschr\as nkt ist. Da $T_1$ ein kompakter Operator ist, gilt (mit Definition der Folgenkompaktheit), dass \(\overline{T_1\overline{K_1^V(0)}}\) kompakt ist. Da stetige Abbildungen Kompaktheit erhalten, ist also auch \(S_1 \overline{T_1\overline{K_1^V(0)}}\) kompakt, d. h. insbesondere totalbeschr\as nkt. Somit gilt (unter Ausnutzung von Bem. \ref{totally_bounded}).
	\[S_1 T_1 \overline{K_1^V(0)} \subseteq S_1\overline{T_1\overline{K_1^V(0)}} \text{ ist totalbeschr\as nkt.}\]
	Betrachte nun \(T_1S_2\), auch hier gen\us gt zu zeigen, dass \(T_1S_2 \overline{K_1^U(0)}\) totalbeschr\as nkt ist. Mit \(\norm {S_2 u} \leq \norm{S_2}\norm{u}\) f\us r alle \(u\in U\) gilt
	\[S_2 \overline{K_1^U(0)} \subseteq \overline{K_{\norm{S_2}}^V(0)} = \norm{S_2} \overline{K_1^V(0)}\;.\]
	Damit gilt nun 
	\[T_1S_2 \overline{K_1^U(0)} \subseteq T_1\norm{S_2} \overline{K_1^V(0)} = \norm{S_2} T_1\overline{K_1^V(0)}\;.\]
	Da $T_1$ kompakt, ist also \(T_1 \overline{K_1^V(0)}\) totalbeschr\as nkt und somit auch \(T_1S_2\overline{K_1^U(0)}\).
\end{proof}
\begin{rem}
	\label{totally_bounded}
	Anders als bei kompakten Mengen, gilt f\us r totalbeschr\as nkte Mengen, dass jede Teilmenge einer totalbeschr\as nkten Menge wieder totalbeschr\as nkt ist.
\end{rem}

\begin{theorem}
	Seien \(V,W\) Banchr\as ume und \(\forall n \in \N: T_n \in \mc K(V,W)\) sowie \(T\in B(V,W)\). Sei \(\norm\cdot\) die Operatornorm und gelte \(\li Tn = T\) bzgl. \(\norm\cdot\). Dann gilt auch \(T\in \mc K(V,W)\).
\end{theorem}

\begin{proof}
	Es g. z. z., dass \(T\overline{K_1^V(0)}\) total beschr\as nkt ist. Sei \(\varepsilon > 0\) fixiert und \(n \in \N\) so, dass \(\norm{T-T_n}<\frac{\varepsilon}{3}\). Da nach Voraussetzung $T_n$ kompakt, ist \(T_n \overline{K_1^V(0)}\) totalbeschr\as nkt, d. h. nach Definition
	\[\exists x_1,\ldots,x_m \in \overline{K_1^V(0)}: T_n \overline{K_1^V(0)} \subseteq \bigcup_{i=1}^m K_{\frac\varepsilon3}(T_n x_i)\;.\]
	Sei nun \(x\in \overline{K_1^V(0)}\), d. h. \(\norm{x}_V \leq 1\). Damit gilt
	\[T_n x \in T_n\overline{K_1^V(0)} \implies \exists i\in \set{1,\ldots,m}: \norm{T_n x - T_n x_i} < \frac{\varepsilon}{3}\;.\]
	Nun l\as sst sich die Dreiecksungleichung anwenden mit (\(\normn{x}V \leq 1, \;\normn{x_i}V \leq 1\)):
	\begin{align*}
		\norm{Tx - Tx_i} &\leq \norm{Tx - T_n x} + \norm{T_n x - T_n x_i} + \norm{T_n x_i - T x_i}\\
		&  = \norm{(T-T_n)x }+ \norm{T_n x - T_n x_i}  + \norm{(T_n -T)x_i}\\
		& < \norm{(T-T_n)x }+ \frac\varepsilon 3 + \norm{(T_n -T)x_i}\\
		&\leq \norm{T-T_n}\normn xV + \frac\varepsilon 3 +\norm{T_n - T} \normn{x_i}V \leq \varepsilon
	\end{align*}
	Somit gilt also 
	\[T\overline{K_1^V(0)} \subseteq \bigcup_{i=1}^m K_\varepsilon (x_i) \iff T\overline{K_1^V(0)} \text{ totalbeschr\as nkt}\]
	d. h. $T$ ist in der Tat kompakt. 
\end{proof}