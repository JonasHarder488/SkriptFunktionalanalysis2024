
\begin{lemma} Sei $S, T, U \in \mc B(H),$ mit $H$ als Hilbertraum und $T = SU = US.$ Dann existiert $T^{-1}$ genau dann, wenn $S^{-1}$ und $U^{-1}$ existieren. 
	
	
	\begin{proof}[Beweis] $(\Leftarrow)$ Ist klar, da \[T^{-1} = S^{-1}U^{-1} = (US)^{-1} = U^{-1}S^{-1}\] \\ \\
		
		$(\Rightarrow)$ Wir zeigen dies über die Kontraposition. Nehme an $S^{-1}$ existiere nicht. \\ \\
		
		\textbf{Fall 1:} Sei \[\ker{S} \not= \set{0}, T = US \implies \ker{T} \supseteq \ker{S} \not= \set{0}\] \\
		
		\textbf{Fall 2:} Sei $\ker{S} = \set{0}.$ Dann kann $S$ aber nicht surjektiv sein. Betrachte:
		
		\[(TH)^{\perp} = (SUH)^{\perp} \supset (SH)^{\perp} \not= \set{0}\] 
		
		\unsure{Weil $H = TH + TH^{\perp}???$}
		
	\end{proof}
	
\end{lemma}


\begin{theorem} \label{lem_inv} Für $T \in \mc B(H), H$ Hilbertraum, gilt für $p \in \C[X]: \sigma(p(T))= p(\sigma(T))$
	
	
	\begin{proof}[Beweis] $(\subseteq)$ Sei $\mu \not= \sigma(p(T)),$ also $p(T)-\mu$ nicht invertierbar. Nach dem Fundamentalsatz der Algebra gibt es ein $c \not= 0, \lambda_1, …, \lambda_n \in \C$ mit \[p(x)- \mu = c(x- \lambda_1)\cdot \dots \cdot (x- \lambda_n) \implies p(T)-\mu = c(T-\lambda_1) \cdot … \cdot (T- \lambda_n)\]
		
		Damit gibt es $\lambda_i (T), i = 1, …, n$ mit $(T- \lambda_i)$ existiert nicht, also $\lambda_i \in \sigma(T).$ Aber $p(\lambda_i)- \mu = 0 \implies \mu = p(\lambda_i).$ \\ \\
		
		$(\supseteq)$ Sei $\lambda \in \sigma(T).$ Es gibt ein $c \not= 0, \lambda_1, …, \lambda_n: p(x)-p(\lambda) = c(x-\lambda_1)\cdot \dots \cdot (x- \lambda_n).$ Sei $x \not= \lambda_i, i = 1, \dots,j, \dots, n,$ dann folgt aus \[0 = c(x- \lambda_1)\cdot \dots \cdot (x- \lambda_j) \cdot \dots \cdot (x- \lambda_n) \implies x= \lambda_j\]
		
		Dann \[p(T)-p(\lambda_j)\text{Id} = cc(x- \lambda_1)\cdot \dots \cdot (x- \lambda_j) \cdot \dots \cdot (x- \lambda_n) = c(x- \lambda_1)\cdot \dots \cdot (x- \lambda_{j+1}) \cdot \dots \cdot (x- \lambda_n) (x- \lambda_j).\] Da aber $(T-\lambda_j)$ nicht existiert existiert durch \ref{lem_inv} auch $(P(T)-p(\lambda)\text{Id})^{-1}$ nicht. Damit $p(\lambda_j \in \sigma(p(T))).$
		
	\end{proof}
	
\end{theorem}


\begin{definition} Sei $T \in \mc B(H), H$ ein Hilbertraum, dann ist der numerische Wertebereich definiert als $W(T) := \set{\ip{x, Tx}: \norm{x} = 1}.$ 
	
\end{definition}


\begin{theorem} \label{num_spec} Sei $T \in \mc B(H),$ dann $\sigma(T) \subseteq \overline{W(T)}.$
	
	\begin{proof}[Beweis] Wir zeigen dies über die Kontraposition. Sei also $\lambda \notin \overline{W(T)},$ dann ist zu zeigen $(T-\lambda\text{Id})^{-1}$ existiert. Damit $d := \text{dist}(\lambda, \overline{w(T)}) > 0$ und $\overline{W(T)} \subseteq \overline{K_{\norm{T}}(0)}$ also folgt Kompaktheit. \\ \\
		
		Sei nun $\norm{x} = 1,$ dann:
		
		\[d \leq \ab{\lambda - \ip{x, Tx}} = \ab{\lambda \ip{x, x} - \ip{x, Tx}} = \ab{\ip{x, (T-\lambda \text{Id})x}} \leq \underbrace{\norm{x}}_{= 1} \norm{(T-\lambda{text{Id}})x}\]
		
		Also gilt für alle $x \in H$ mit $norm{x} = 1.$ Damit aber auch für alle $x \in H\setminus \set{0}$ (Für $\set{0}$ trivialerweise.): \[\]
		
		\[\norm{(T-\lambda \text{Id})x} = \norm{x} \norm{(T-\lambda \text{Id})\frac{x}{\norm{x}}} \geq d \norm{x} \implies \ker{T - \lambda \text{Id}} = \set{0}\] Damit haben wir die Injektivität. \unsure{Wie geht es weiter???}%
		
		Außerdem ist $(T- \lambda \text{Id})H$ abgeschlossen \unsure{Warum?}.
		
		Sei nun $y_n = (T-\lambda \text{Id}x_n) \xrightarrow{n \to \infty} y.$ Wenn $(y_n)_{n \in \N}$ eine Cauchy-Folge ist, so auch $(x_n)_{n \in \N}.$ \\ \\
		
		Sei nun $(T- \lambda \text{Id})H$ nicht dicht, d.h es gibt $0 \not= x \in ((T-\lambda \text{Id})H)^{\perp}, \norm{x} = 0.$ Dann folgt: \[0 = \ip{x, (T- \lambda \text{Id})x} = \ip{x, Tx}-\lambda \implies \lambda \in \overline{W(T)}\]
		
		und damit ein Widerspruch. 
		
	\end{proof}
	
\end{theorem}


\begin{theorem} Sei $T \in \mc B(H)$ selbstadjungiert mit $T = T^*,$ dann ist $\sigma(T) \subseteq [- \norm{T}, \norm{T}] \subseteq \R.$
	
	
	\begin{proof}[Beweis] Es gilt mit \ref{num_spec} $\sigma{T} \subseteq \overline{W(T)} \subseteq [-\norm{T}, \norm{T}] \subseteq \R,$ da $\ip{x, Tx} = \ip{Tx, x} = \overline{x, Tx} \in \R.$ Des Weiteren mit $\norm{x} = 1:$ \[\ab{\ip{x, Tx}} \leq \norm{T}\norm{x}^2 = \norm{T}\]
		
	\end{proof}
	
\end{theorem}


\section{Der Satz von Stone-Weierstraß}

\begin{theorem} Sei $f \in C^0_{\R}[a, b]$ dann gibt es eine Folge von Polynomen $(p_n)_{n \in \N}, p_n \in \R[X]$ mit $p_n \xrightarrow{\norm{\cdot}_{\infty}, n \to \infty}f.$
	
	\begin{proof}[Beweis] Ohne Beweis.
		
	\end{proof}
	
\end{theorem}


\begin{theorem} \label{algebraquatsch} Sei $\emptyset \not= K$ kompakt, $\mc A \subseteq C^0(K, \R)$ mit 
	
	\begin{enumerate}
		
		\item $\forall f, g \in \mc A, \lambda \in \C: \lambda f +g \in \mc A, fg \in \mc A$
		
		\item $\mc A$ abgeschlossen
		
		\item $\forall s \not= t \in K \exists f \in \mc A: f(s) \not= f(t)$
		
		\item $1 \in \mc A$
		
	\end{enumerate}
	
	Dann ist $\mc A = C^0[K, \R]$
	
	\begin{proof}[Beweis] Wir zeigen $f \in \mc A, 0 \leq f \implies \sqrt{f} \in \mc A.$ Sei o.B.d.A $f \leq 1,$ dann existiert $g_n := \sqrt{\frac{1}{n}+(1- \frac{1}{n})f}$ für alle $n \in \N.$
		
		\unsure{Warum gilt hier Gleichheit?} Und dies ist gleich \[g_n = \frac{1}{n}+ (1- \frac{1}{n})f \leq \frac{1}{n}+(1- \frac{1}{n}) = 1 \implies \norm{g_n}_{infty} = \sqrt{1} = 1\] Aufgrund der Abgeschlossenheit haben wir nun: \[\lim_{n \to \infty} g_n = \sqrt{f} \in \mc A.\] \\ \\
		
		Wir erhalten nun folgendes geschenkt:
		
		\begin{enumerate}
			
			\item $\forall f \in \mc A: \ab{f} = \sqrt{f^2} \in \mc A.$
			\item $\forall f, g \in \mc A: f \lor g = \max(f, g) = \frac{1}{2}(f+g+ \ab{f-g}) \in \mc A$
			\item $\forall f, g \in \mc A: f \land g = \min(f, g) = \max{\set{-f, -g}}$
			
		\end{enumerate}
		
		Sei nun $f \in C^0(K, \R), \varepsilon > 0.$ Für alle $s \not= t \in \mc A$ gibt es ein $h \in h(s) \not= h(t),$ sei nun \[f_{s, t}: K \to \R, f_s,t(v) = f(t)+(f(s)+f(t))\frac{h(v)-h(t)}{h(s)-h(t)} \implies f_{s, t} \in \mc A\] Außerdem $f_{s, t}(s) = f(s), f_{s, t}(t) = f(t).$ \\ \\
		
		Definiere nun $U_t = \set{v \in K: f_{s, t}(v) < f(v)+\varepsilon}.$ Dann ist $U_t$ offen. Setze nun $s \in K$ fix. Aufgrund der Kompaktheit gilt: \[\exists I = \set{t_1, …, t_m}: \Cup_{t \in I}U_t = K\] 
		
		Wir setzen $h_s := \cap_{t \in I}f_{s, t} \in \mc A$ (und schneiden damit nur über endlich viele.) Wir wissen $h_s(s)=f(s)$ und $\forall v \in K: h_s(v) \subset f(v)+\varepsilon.$
		
		Für alle $s \in K$ ist $v_s = \set{v \in K, f(v)- \varepsilon \subset h_s(v)}$ offen. Es folgt mit Kompaktheit wieder $K = \cup_{j=1}^n v_{s_j},$ und setzte $g_i = \cup_{j=1}^n h_{s_j} \in \mc A$ (wieder nur endlich viele), also gilt für alle $v \in K: g(v) < f(v)+\varepsilon$ und $g(v) > f(v) -\varepsilon.$ Folglich $\norm{g-f}_{\infty} < \varepsilon.$ Damit haben wir die Dichtheit gewonnen. Da alle Grenzwerte aufgrund der Abgeschlossenheit in $\mc A$ enthalten sind, erhalten wir die Behauptung.
		
	\end{proof}
	
\end{theorem}


\begin{ex} Sei $K = [a, b]$ und betrachte $\mc A = \overline{\R[X]}.$ Dann folgt mit \ref{algebraquatsch} der Satz von Weierstraß. 
	
\end{ex}


\begin{ex} Sei $K = S^1, C^0(S^1) = \set{f: \R \to \R \text{ stetig, }2\pi \text{ periodisch}}$ und \\\(\mc A = \overline{\varepsilon_0 + \sum_{n = 1}^{\infty}a_n \sin{(nx)} + b_n \cos{(nx)}}\)
	
\end{ex}

\section{Der stetige Funktionalkalk\us l}
\begin{theorem}[Stetiger Funktionalkalk\us l]
	\label{stetiger_kalkuel}
	Sei $H$ ein Hilbertraum, \(T\in \B(H)\) selbstadjungiert. Dann existiert genau eine Isometrie 
	\[\varphi_T: C^0(\sigma(T),\C) \to \B(H)\]
	mit den folgenden Eigenschaften:
	\begin{enumerate}
		\item $\varphi_T$ ist ein involutiver Algebrenmorphismus, d. h. die Abbildung ist
		\begin{itemize}
		\item linear:\;\; \(\forall f, g \in C^0(\sigma(T)), \;\lambda \in \C: \varphi_T( \lambda f+g)  = \lambda \varphi_T(f) + \varphi_T(g)\)
		\item multiplikativ: \;\;\(\forall f, g \in C^0(\sigma(T)): \varphi_T(f\cdot g) = \varphi_T(f)\cdot\varphi_T(g) = \varphi_T(g) \cdot \varphi_T(f)\)
		\item involutiv: \;\; \(\forall f \in C^0(\sigma(T)): \varphi_T(\overline{f}) = \varphi_T(f)^*\)
		\end{itemize}
		\item \(\varphi_T(1) = \I\) und \(\varphi_T(\text{id}) = T\), wobei \(\text{id}(x) = x\), \(1(x) = 1\).
	\end{enumerate}
\end{theorem}

\begin{rem}
	Wir schreiben \(f(T) := \varphi_T(f)\).
\end{rem}

\begin{proof}[Beweis]
	\textbf{Schritt 1:} Wir definieren:
	\[A_0 := \set{f \in C^0(\sigma(T)) \;\Big\vert\; \exists p \in \C[X]: f(x) = p(x):= \sum_{n=0}^N a_n \cdot X^n \;\forall x \in \sigma(T) }\;.\]
	Somit k\os nnen wir f\us r alle  \(f \in A_0:\) \(\varphi_0 (f) = p(T)\) sowie \(\varphi_0(f)^* = \overline{p}(T)\) setzen (mit $p$ zu $f$ wie in der Definition), wobei:
	\[p(T) := \sum_{n=0}^N a_n T^n\; \text{ und }\;\overline{p}(X) := \sum_{n=0}^N \overline{a_n}X^n \;.\]
	Dabei nutzen wir \((T^n)^* = (T^*)^n = T^n\) aus.\\
	\textbf{Schritt 2:} Nach Satz \textit{[\ldots]} ist $A_0$ nach Konstruktion dicht in \(C^0(\sigma(T))\).\\
	\unsure{Label fehlt (Satz 8.11 in Vorlesung)}
	\textbf{Schritt 3:} Es gilt f\us r alle \(f, g \in A_0\) mit $p, q$ wie in der Definition
	\[ \varphi_0(fg) = pq(T) =  p(T) q(T) = \varphi_0(f)\varphi_0(g)\;.\]
	Weiterhin ist die Linearit\as t offensichtich erf\us llt. \\
	\textbf{Schritt 4:} Folglich gilt f\us r alle $f \in A_0$ mit $p$ wie in der Definition:
	\begin{align*}
	\norm{\varphi_0(f)}^2 &= \norm{\varphi_0(f)^* \varphi_0(f)} = \norm{\varphi_0(p \overline{p})} \overset{(*)}{=} r(\varphi_0(\overline p p )) = \sup\set{\ab{\lambda} \;\vert\; \lambda \in \sigma(\varphi_0(\overline p p ))} \\
	&  = \sup\set{\ab{\lambda} \;\vert\; \lambda \in \sigma(\overline p p (T))} \overline{(**)}{=} \sup\set{\ab{\lambda} \;\vert\; \lambda \in \overline p p (\sigma(T))}\\& = \sup\set{\ab{p(\lambda)}^2 \;\vert\; \lambda \in \sigma(T)}  = \norms p ^2
	\end{align*}
	Dabei haben wir im Schritt \((*)\) ausgenutzt, dass \(\varphi_0(\overline p p )\) selbstadjungiert ist und im Schritt \((**)\) Satz \textit{[...]}. Somit ist $\varphi_0$ in der Tat eine Isometrie. \\
	\unsure{Label fehlt.}
	\textbf{Schritt 5:} Sei \(f\in C^0(\sigma(T))\). Nach dem Satz von Weierstra\ss{} existiert eine Folge von Polynomen \(p_n \in A_0\) mit \(\li pn = f\) (bzgl. \(\norms\cdot\)). Folglich ist \(\seq pn \) eine Cauchyfolge und aufgrund der Isometrieeigenschaft auch \((\varphi_0(p_n))_{n\in\N}\). Somit konvergiert letztere Folge und wir setzen \(\varphi_T(f)\) auf den Grenzwert.\\
	\textbf{Schritt 6:} (Linearit\as t) Seien \(f,g \in C^0(\sigma(T))\), mit entsprechenden Polynomfolgen in $A_0$: 
	\[\li pn = f,\; \li qn = g \;\text{(bzgl. $\norms\cdot$)}\;.\]
	Es gilt \(\lime{n}{p_n + q_n}= f+g\) und folglich
	\[\varphi_T(f+g) = \lime{n}{\varphi_0(p_n + q_n)} = \lime{n}{\varphi_0(p_n)} + \lime{n}{\varphi_0(q_n)} = \varphi_T(f) + \varphi_T(g)\;.\]
	\textbf{Schritt 7:} (Wohldefiniertheit) Sei \(f\in C^0(\sigma(T)))\) mit \(\li pn = f\) bzgl. \(\norms\cdot\). Dann gilt 
	\[\norm{\varphi_T(f)} = \norm{\lime{n}{\varphi_T(p_n)}} = \lime{n}{\norm{\varphi_0(p_n)}} = \lime{n}{\norms{p_n}} \]
	\textbf{Schritt 8:} (Eindeutigkeit) Sei \(\tilde\varphi\) eine andere Isometrie mit den obigen Eigenschaften. Dann gilt \(\forall f \in A_0\) (mit $p$ wie in der Definition):
	\[\tilde \varphi(f) = \varphi_0(f) = p(T) = \varphi_T(f)\;.\]
	Betrachte nun allgemein \(g \in C^0(\sigma(T))\) mit entsprechender Polynomfolge \(\li qn = g\) (bzgl. $\norms\cdot$). Dann gilt
	\[\lime{n}{\varphi_0(q_n)} = \tilde\varphi(g) \;\text{ und }\; \lime{n}{\varphi_0(q_n)} =\varphi_T(g) \implies \tilde\varphi(g) = \varphi_T(g)\;.\]
	\end{proof}
	\begin{rem}[Stetiger Funktionalkalk\us l f\us r normale Operatoren]
	  Der Beweis l\as sst sich analog f\us r normale Operatoren durchf\us hren. Dabei betrachten wir in der Definition von $A_0$ jedoch 
	  \[\C[X, \overline X] \;\text{ und somit z. B. }\; A_0 \ni p(X, \overline X) = X\overline X = \overline X X \implies \varphi_T(p) = TT^* = T^*T\;.\]
	  F\us r normale Operatoren besitzt die Isometrie die zus\as tzliche Eigenschaft
	  \(\varphi_T(\overline {\text{id}}) = T^*\;.\)
	  \end{rem}
	\begin{lemma}
		\label{spektrum_normal}
		Sei \(T\in \B(H)\) normal, \(\lambda \in \sigma(T)\). Dann existiert \(\seq xn \subseteq H^\N\) mit 
		\[\forall n \in \N: \; \norm{x_n} = 1 \;\text{ und }\;  \lime{n}{\norm{Tx_n - \lambda x_n}} = 0\;.\]
	\end{lemma}
	\begin{proof}[Beweis]
		Wir benutzen Kontraposition. Angenommen
		\[\exists \delta > 0 \;\forall x \in H, \norm x = 1: \norm{Tx - \lambda x } \geq \delta\;.\]
		Somit folgt
		\[\forall x \in H: \norm{(T-\lambda \I)x} \geq \delta \norm x \implies  T -\lambda \I \;\text{ injektiv} \implies (T-\lambda\I)H \;\text{ abgeschlossen}\;.\]
		Dabei erhalten wir die letzte Folgerung aus Satz \ref{Offene_Abb}.\\
		Sei nun \(H \ni u \perp (T-\lambda\I)H\), d. h. 
		\[\forall x\in H\setminus\set{0}:\; 0 = \ip{u, (T-\lambda\I)x} \iff 0  = \ip{(T^* - \overline\lambda \I)u,x}  \implies  (T^* -\overline \lambda I) u = 0\;.\]
		Dabei haben wir ausgenutzt, dass gilt:
		\[ \ip{u, (T-\lambda\I)x}  = \ip{u,Tx} - \lambda \ip{u,x} = \ip{T^* u, x} - \ip{\overline\lambda u,x}\;. \]
		Folglich gilt:
		\begin{align*}0 &= \ip{(T^* -\overline \lambda I) u,(T^* -\overline \lambda I) u} = \ip{u, (T-\lambda\I)(T^* -\overline \lambda I) u} = \ip{u, (T^* -\overline \lambda I)(T-\lambda\I) u} \\
		&= \norm{(T-\lambda\I)u}^2 \implies u=0 \implies \lambda \in \varrho(T)\;.
		\end{align*}
		Dabei haben wir \(u=0\) aus der Injektivit\as t von \(T-\lambda\I\) gefolgert. Damit ist \(T-\lambda\I\) auch surjektiv und es folgt die Behauptung.
	\end{proof}
	
	\begin{theorem}\label{eigenschaften_stetiger_kalkuel}
		Sei $H$ ein Hilbertraum, \(T\in \B(H)\) normal.  Dann gilt f\us r \(f\in C^0(\varrho(T))\):
		\begin{enumerate}
			\item \(\norm{f(T)}_{\B(H)} = \norms{f} = \sup\set{\ab{f(\lambda)}\;\vert\;\lambda \in \sigma(T)}\) \label{eigenschaften_stetiger_kalkuel_1}
			\item Falls \(x\in H\setminus\set{0}\), \(T\) selbstadjungiert sowie \(\lambda \in \R\) die Eigenwertgleichung \(Tx = \lambda x\) erf\us llen, so gilt \(f(T)x = f(\lambda)x\)\;.
			\item Spektralabbildungssatz: Es gilt \(\sigma(f(T)) = f(\sigma(T)) = \set{f(\lambda)\;\vert\;\lambda \in \sigma(T)}\).
		\end{enumerate}
	\end{theorem}
	\begin{proof}[Beweis]
		\textbf{Zu 1.)} Dies folgt sofort aus Schritt 4, Satz \ref{stetiger_kalkuel}.\\
		\textbf{Zu 2.)} Nach Konstruktion im Beweis von Satz \ref{stetiger_kalkuel} ist bekannt:
		\[\forall p \in \C[X]: p(T)x = p(\lambda)x\;.\]
		Sei nun eine Polynomfolge \(\seq pn\) gegeben mit \(\li pn = f\) (bzgl. \(\norms\cdot\)). Mit \ref{eigenschaften_stetiger_kalkuel_1} folgt:
		\begin{align*}
		&\lime{n}{p_n(T)} = f(T)\;\text{ (bzgl. $\norm{\cdot}_{\B(H)}$)} \implies \lime{n}{p_n(T)x} = f(T)x \;\text{ bzgl. $\norm{\cdot}_{H}$} \\
		& \text{ mit }\lime{n}{p_n(T)x} = \lime{n}{p_n(\lambda)x} = f(\lambda)x \implies f(T)x = f(\lambda)x\;.
		\end{align*}
		\textbf{Zu 3.)} \textbf{($\subseteq$)} Es g. z. z. 
		\[\forall \mu \in \sigma(f(T)) \;\exists \lambda \in \sigma(T): \mu = f(\lambda)\;.\]
		Wir benutzen Kontraposition. Angenommen \(\mu \not\in f(\sigma(T))\). Dann folgt
		\[g := \frac{1}{f-\mu} \in C^0(\sigma(T)) \;\text{ und }\; g(f-\mu) = (f-\mu)g = 1 \in C^0(\sigma(T))\;.\]
		Nach Satz \ref{stetiger_kalkuel} gilt somit
		\[g(T)(f(T) -\mu\I) = (f(T) -\mu\I)g(T) = \I \implies (f(T) -\mu\I)^{-1}\;\text{ existiert.}\]
		Somit gilt in der Tat \(\mu\not\in \sigma(f(T))\).\\
		\textbf{($\subseteq$)} Sei \(\mu = f(\lambda)\) f\us r ein \(\lambda \in \sigma(T)\). W\as hle Polynomfolge \(\seq gn \subseteq \C[X,\overline X]^\N\) mit \(\norms{f - g_n} \leq \frac1n\). Somit gilt insbesondere \(\ab{f(\lambda) - g_n(\lambda)} \leq \frac 1n\). Unter Ausnutzung von \(\sigma(g_n(T)) = g_n(\sigma(T))\) (Bew. von Satz \ref{stetiger_kalkuel}) existiert also (nach Lemma \ref{spektrum_normal}) \(\seq xn \subseteq H^\N\) mit 
		\[\forall n\in\N: \norm{x_n} = 1 \;\text{ und } \; \norm{(g_n(T) - g_n(\lambda)\I)(x_n)}\leq \frac 1n\;.\]
		Damit gilt
		\begin{align*}
			\norm{(f(T) -\mu\I) x_n} &\leq \norm{(f(T) - g_n(T))x_n} + \norm{(g_n(T) - g_n(\lambda)\I)x_n} \\
			&+ \norm{(g_n(\lambda)\I - f(\lambda)\I)x_n} \overset{n\to\infty}{\longrightarrow} 0\;.
			\end{align*}
			Somit gilt in der Tat \(\mu\in\sigma(f(T))\).
	\end{proof}
	
	\section{Positive Operatoren und die Spurklasse}
	
	\begin{definition}
		Sei $H$ ein Hilbertraum, dann bez. \(T\in\B(H)\) \textit{positiv} \(:\iff T = T^*\) und \(\sigma(T) \geq 0\) (d. h. \(\sigma(T) \subseteq [0,\infty)\)). Wir schreiben dann auch \(T \geq 0\).
	\end{definition}
	
	\begin{theorem}
		Sei \(T\in\B(H)\) selbstadjungiert, dann gilt
		\begin{enumerate}
			\item \(\ab{T} \geq 0\)
			\item \(T\geq 0 \implies \exists S \geq 0: T = S^2\) ($S = \sqrt{T}$)
		\end{enumerate}
		\begin{proof}[Beweis]
			Dies folgt aus den S\as tzen \ref{stetiger_kalkuel} und \ref{eigenschaften_stetiger_kalkuel}, da \(\ab\cdot\) und \(\sqrt\cdot\) stetige Funktionen sind.
		\end{proof}
	\end{theorem}
	
	\begin{theorem}
		Sei \(T\in \B(H)\), wobei $H$ ein Hilbertraum \us ber $\C$. Dann gilt
		\[T \geq 0 \iff \forall x \in H: \ip{x,Tx} \geq 0 \;.\]
	\end{theorem}
	\begin{proof}[Beweis]
		\textbf{($\Longrightarrow$)} Es gilt
		\[\ip{x,Tx} = \ip{x,\sqrt{T}\sqrt{T} x} = \ip{\sqrt{T}x, \sqrt{T}x} \geq 0\;.\]
		\textbf{($\Longleftarrow$)}
		Nach Definition des numerischen Wertebereichs gilt
		\[\sigma(T) \subseteq \overline{W(T)} = \overline{\set{\ip{x, Tx}\;\vert\;\norm{x} = 1}} \subseteq [0,\infty)\;.\]
	\end{proof}